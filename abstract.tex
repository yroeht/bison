%\documentclass[a4paper,11pt,final]{article}
% Pour une impression recto verso, utilisez plutôt ce documentclass :
\documentclass[a4paper,11pt,final]{article}

\usepackage[english,francais]{babel}
\usepackage[utf8]{inputenc}
\usepackage[T1]{fontenc}
\usepackage[pdftex]{graphicx}
\usepackage{setspace}
\usepackage{hyperref}
\usepackage[french]{varioref}
\usepackage{changepage} %for changing margins in appendices
\usepackage{gantt}
\usepackage{fancyhdr}

\newcommand{\reporttitle}{Internship Report Abstract}
\newcommand{\reportauthor}{Théophile \textsc{Ranquet}}
\newcommand{\reportsubject}{Fourth year internship}
\newcommand{\HRule}{\rule{\linewidth}{0.5mm}}
\setlength{\parskip}{1ex} % Espace entre les paragraphes
\newcommand{\ts}{\textsuperscript}

\hypersetup{pdftitle={\reporttitle},
            pdfauthor={\reportauthor},
            pdfsubject={\reportsubject},
            pdfkeywords={rapport} {vos} {mots} {clés}
}

\begin{document}
Théophile Ranquet
  id: 64201
\begin{center}
  \begin{minipage}[t]{0.48\textwidth}
    \begin{flushleft}
      \includegraphics [width=50mm]{images/logo-epita.png} \\[0.5cm]
      \begin{spacing}{1.5}
      \end{spacing}
    \end{flushleft}
  \end{minipage}
  \begin{minipage}[t]{0.48\textwidth}
    \begin{flushright}
      \includegraphics [width=35mm]{images/logo-lrde.png} \\[0.5cm]
    \end{flushright}
  \end{minipage} \\[0.5cm]

    \textsc{\Large \reportsubject}\\[0.5cm]
    \large{September 7\ts{th}, 2012 - January 31\ts{st}, 2013}\\[0.1cm]
    \large{EPITA's Research and Development Laboratory (LRDE)}\\[0.3cm]
    \HRule \\[0.4cm]
    {\huge \bfseries \reporttitle}\\[00.4cm]
    \HRule \\[0.2cm]

    \large{Contribution to the development of GNU Bison}\\[0.2cm]

    \begin{minipage}[t]{0.3\textwidth}
      \begin{flushleft} \large
        \emph{Author :}\\
        \reportauthor\\
      \end{flushleft}
    \end{minipage}
    \begin{minipage}[t]{0.6\textwidth}
      \begin{flushright} \large
        \emph{Supervisor :} \\
        Akim \textsc{Demaille} (maintainer)\\
      \end{flushright}
    \end{minipage}
  \end{center}

  \sloppy
  \pagenumbering{arabic}
  \setcounter{page}{1}
  \section*{Introduction}
  \addcontentsline{toc}{section}{Introduction}

  This report will give a brief but comprehensive overview of my five months
  internship at EPITA's Research and Development Laboratory (LRDE).  My work
  was on Bison, which is part of the GNU project.  Therefore, this report will
  not contain informations pertaining to the LRDE's work \textit{per se}.

  This abstract will start with a reminder of who and what stands behind the
  entities I worked with (LRDE, GNU). It will then proceed to give a rough
  outline of what the software I worked on (Bison) is, the state in which it
  was when I arrived, and what my contributions to it have been. Finally, it
  well give an appreciation of the work that was done, and recap the skills I
  have developed.

  \section*{Presentation of the laboratory}

  The EPITA Research and Development Laboratory was created at the
  beginning of year 1998 at EPITA, a private university in computer science
  located in the south of Paris, France. The different skills of LRDE's members
  cover various fields of computer science, including (but not limited to):

  \begin{itemize}
    \item Image and Signal Processing
    \item Compilation
    \item Automata Theory
    \item Metaprogramming
  \end{itemize}

  The LRDE's permanent staff is composed of eight computer researchers, several
  of them being involved in the free software community (XEmacs, autotools,
  Bison, and others).

  \section*{Presentation of the project: GNU Bison}

  The GNU Project is a free software, mass collaboration project,
  announced on 27 September 1983, by Richard Stallman at MIT\@. Its aim is to
  give computer users freedom and control in their use of their computers and
  computing devices, by collaboratively developing and providing software that
  is based on the following freedom rights: users are free to run the software,
  share it (copy, distribute), study it and modify it.

  GNU bison, commonly known as Bison, is a parser generator that is part of the
  GNU Project. Bison reads a specification of a context-free language, warns
  about any parsing ambiguities, and generates a parser (either in C, C++, or
  Java) which reads sequences of tokens and decides whether the sequence
  conforms to the syntax specified by the grammar. Bison by default generates
  LALR parsers but can also create GLR parsers.

  The internship was supervised by Akim Demaille. Akim is the current
  maintainer of Bison, alongside Joel E. Denny, and Paul Eggert. He is also a
  permanent member of the LRDE, and a teacher at EPITA, for Language Theory,
  Compiler Construction, and Formal Logic.

  The activity from other contributors to GNU Bison was pretty low during my
  internship, so the "team" I was a part of was basically just Akim and I.

  \section*{The existing state of affairs}

  Bison started in 1986 as the GNU implementation of the old YACC (developed by
  Stephen C. Johnson at AT\&T Corporation for the Unix operating system in
  1970), by Richard M. Stallman.

  In a Nutshell, GNU Bison has had 5,201 commits made by 33 contributors over
  26 years, representing 33,650 lines of code. The version number at my arrival
  was 2.6.2, so the project was overall already very mature. There were no
  longstanding bugs to fix, just the perspective of a few potentially nice
  feature additions

  Bison itself is written in C90 (70\%); the skeletons for the generated
  parsers are processed using GNU M4 macros, and contain (depending on the
  skeleton) C, C++, or Java; the rest of the project is autoconf script (m4)
  for the testsuite; there are also minor bits of XSLT for generating nice
  graphical representations of the parser automatons, and Perl for benchmarking
  tools, for example.

  \section*{Description and timeline of the internship}

  The first task that I was assigned was to continue the work that a previous
  intern (Victor Santet) had started. Historically, Bison had errors and
  warnings, and that was all. Victor introduced the distinction between
  warning classes, \textit{à la} GCC\@. He also introduced the means to treat
  warnings as errors, with a \texttt{-Werror} option. My improvements include:
  \begin{itemize}
    \item introducing a much more specific \texttt{-Werror=CATEGORY} option
    \item rehauling other aspects of error reports, such as indentation and
      prefixing
    \item introducing \textit{caret diagnostics}
  \end{itemize}

  The second task was improving the graphical representation of the generated
  \textit{shift/reduce} automatons, to make it more pleasant for the user (more
  eye-candy) and more useful: only the \textit{shifts} were represented, the
  \textit{reductions} were not.

  All of these previous tasks were C (the last one had a part of it in XSLT
  though). This was the first half of the internship.

  The second half was all about the skeletons of Bison: C and C++ in a M4-heavy
  environment. I started by implementing a longstanding feature request, and I
  then moved on to fixing a few things which led me to changing the
  representations of the parser's internal symbols, which was a delicate object
  orientated design problem.

  The timeline of the internship is summarized in the following Gantt chart and
  table:

  \begin{adjustwidth}{-4cm}{-2cm}
  \begin{gantt}[xunitlength=0.7cm,fontsize=\small,titlefontsize=\small,drawledgerline=true]{14}{20}
    \begin{ganttitle}
      \titleelement{2012}{16}
      \titleelement{2013}{4}
    \end{ganttitle}
    \begin{ganttitle}
      \numtitle{9}{1}{12}{4}
      \numtitle{1}{1}{1}{4}
    \end{ganttitle}
    \begin{ganttitle}
      \numtitle{1}{7}{22}{1}
      \numtitle{1}{7}{22}{1}
      \numtitle{1}{7}{22}{1}
      \numtitle{1}{7}{22}{1}
      \numtitle{1}{7}{22}{1}
    \end{ganttitle}
    \ganttbar{[1]}{1}{4}
    \ganttbarcon{[2]}{10}{2}
    \ganttbar{[3]}{5}{2}
    \ganttbarcon{[4]}{7}{4}
    \ganttbar{[5]}{10}{2}
    \ganttmilestonecon{version 2.7}{14}
    \ganttbar{[6]}{13}{2}
    \ganttbar{[7]}{15}{2}
    \ganttbarcon{[8]}{17}{3}
  \end{gantt}
  \end{adjustwidth}

  \begin{adjustwidth}{-2cm}{-2cm}
    \begin{center}
      \begin{tabular}{| c | l | l | l |}
        \hline
        Index & Period & Task & Technology \\
        \hline
        1 & from 9/8 to 10/8 & implement warnings as errors & C \\
        2 & from 11/15 to 12/1 & implement caret diagnostics & C \\
        3 & from 10/8 to 10/22 & show reductions on graph visualization file &
        C \\
        4 & from 10/22 to 11/22 & replicate [3] on xml output & XSLT \\
        5 & from 10/15 to 12/1 & implement a new parser purity level & M4 \\
        6 & from 12/8 to 12/22 & hunt some leaks & C \\
        7 & from 12/22 to 1/8 & minor changer to skeletons & M4 / C++ \\
        8 & from 1/8 to 1/31 & change the symbols implementation & C++ \\
        \hline
      \end{tabular}
    \end{center}
  \end{adjustwidth}

  \section*{Global appreciation}

  The internship was most pleasant. Akim taught me many things, including a lot
  of good practices in the world of GNU (coding standards, Git commit messages,
  workflow, etc.) which, I am sure, will come in handy whatever I work on next.
  The other contributors were very kind as well, and answered my questions
  anytime I asked them. All of our discussions is publicly available on the
  project's mailing-list archives. Indeed, contributing to free software
  implies working with people from all over the globe, and thus the
  conversations are done by email.

  My workplace, the LRDE, is great to spend some time with bright minds, and I
  would definitely recommend it. I was seated rather close to Akim, so he was
  constantly looking over me, making sure I was doing good progress, but also
  just chatting and exchanging thoughts. Everyone in the laboratory is on
  first-name basis.

  \section*{Skills acquired}

  \begin{itemize}
    \item problem solving: ability to do much with limited tools, ability to
      introduce a new feature without overhauling the existing features
    \item extensive knowledge of the Git versioning system
    \item good practice of commits using the Git versioning system for
      collaborative work.
    \item better knowledge of coding standards (especially GNU)
    \item practice of technical English (mailing-lists, documentation, etc.)
    \item improved knowledge of C programming (\textit{obstacks}, etc.)
    \item improved knowledge of Graphviz DOT language (for graphs)
    \item learned XSL Transformation language
    \item learned GNU M4 macro language
    \item practiced maintaining an actual testsuite
  \end{itemize}

  \section*{Conclusion}

  This internship was a great opportunity for me. I have a vivid interest for
  fundamental computer science aspects such as Language Theory, and compiler
  implementation. 

  I am very supportive of free software, and I like C, C++, and UNIX
  programming in general. These five months I spent working in that world
  confirmed my desire to continue in that direction.

  There were three huge positive points to the internship. Firstly, I got to
  finally contribute to the GNU project. Secondly, I got to learn M4, and
  improve my C, which are awesome languages to work with. And lastly, most
  important, I got to work with fabulous people.

  The only negative point was the under-average salary, but I didn't care much
  for it.
\end{document}

