%\documentclass[a4paper,11pt,final]{article}
% Pour une impression recto verso, utilisez plutôt ce documentclass :
\documentclass[a4paper,11pt,twoside,final]{article}

\usepackage[english,francais]{babel}
\usepackage[utf8]{inputenc}
\usepackage[T1]{fontenc}
\usepackage[pdftex]{graphicx}
\usepackage{setspace}
\usepackage{hyperref}
\usepackage[french]{varioref}
\usepackage{changepage} %for changing margins in appendices
\usepackage{gantt}

\usepackage{fancyhdr}
\usepackage{lastpage}
\pagestyle{fancy}

\newcommand{\reporttitle}{Rapport de stage}     % Titre
\newcommand{\reportauthor}{Théophile \textsc{Ranquet}} % Auteur
\newcommand{\reportsubject}{Stage de fin de Tronc Commun} % Sujet
\newcommand{\HRule}{\rule{\linewidth}{0.5mm}}
\setlength{\parskip}{1ex} % Espace entre les paragraphes

\hypersetup{pdftitle={\reporttitle},
            pdfauthor={\reportauthor},
            pdfsubject={\reportsubject},
            pdfkeywords={rapport} {vos} {mots} {clés}
}

\begin{document}
  \pagenumbering{roman}
  % Inspiré de http://en.wikibooks.org/wiki/LaTeX/Title_Creation

\begin{titlepage}

\begin{center}
\begin{minipage}[t]{0.48\textwidth}
  \begin{flushleft}
    \includegraphics [width=50mm]{images/logo-epita.png} \\[0.5cm]
    \begin{spacing}{1.5}
    \end{spacing}
  \end{flushleft}
\end{minipage}
\begin{minipage}[t]{0.48\textwidth}
  \begin{flushright}
    \includegraphics [width=35mm]{images/logo-lrde.png} \\[0.5cm]
  \end{flushright}
\end{minipage} \\[1.5cm]

\textsc{\Large \reportsubject}\\[0.5cm]
{\large Du 7 septembre 2012 au 31 janvier 2013}\\[0.1cm]
{\large Au Laboratoire de Recherche et de Développement de l'EPITA}\\[0.5cm]
\HRule \\[0.4cm]
{\huge \bfseries \reporttitle}\\[0.4cm]
\HRule \\[1.5cm]

{\large Contribution aux développements de GNU Bison et Tiger}\\[1.0cm]

\begin{minipage}[t]{0.3\textwidth}
  \begin{flushleft} \large
    \emph{Auteur :}\\
    \reportauthor
  \end{flushleft}
\end{minipage}
\begin{minipage}[t]{0.6\textwidth}
  \begin{flushright} \large
    \emph{Responsables :} \\
    M.~Akim \textsc{Demaille} \\
    M.~Roland \textsc{Levillain}
  \end{flushright}
\end{minipage}

\vfill

{\large 29 janvier 2013}

\end{center}

\end{titlepage}

  \cleardoublepage % Dans le cas du recto verso, ajoute une page blanche si besoin
  \tableofcontents % Table des matières
  \sloppy          % Justification moins stricte : des mots ne dépasseront pas des paragraphes
  \cleardoublepage

  \pagenumbering{arabic}
  \setcounter{page}{1}
  \section*{Introduction} % Pas de numérotation
  \addcontentsline{toc}{section}{Introduction} % Ajout dans la table des matières

  Le stage de 2ème année du cycle d'ingénieur de l'EPITA, aussi appelé stage
  de fin de Tronc Commun, est une expérience critique dans le parcours de
  chacun. On pourrait en dire autant de tout stage, mais celui-ci est le
  dernier avant le stage de fin d'études, qui est un peu différent des autres
  en cela qu'il constitue une sorte de « période d'essai » par les entreprises
  et débouche dans 90\% des cas (chiffres officiels de l'école en 2010) sur une
  embauche (CDI, etc.). Ce stage de fin de tronc commun est donc de fait la
  dernière occasion de découvrir un potentiel métier avant la sortie de
  l'école, et il n'est pas rare que les étudiants choisissent de passer ces
  cinq mois dans un environnement qui n'est pas forcément celui auquel ils se
  destinaient pour leur avenir proche à la fin de leurs études, dans le but de
  s'assurer qu'ils ont fait le bon choix, ou éventuellement de voir si un coup
  de cœur ne les ferait pas changer d'avis.

  Ce stage se place avant les deux semestres de spécialisation des étudiants de
  l'EPITA\@. En ce qui me concerne, j'envisageais avant ce stage de m'orienter
  vers la majeure « Système, Réseau et Sécurité » (SRS). Du coup, il me fallait
  trouver un stage qui me plaise, mais qui me donne une expérience que je
  n'aurai pas forcément l'occasion de reproduire dans un an. Alliant mes
  intérêts pour la programmation Unix en C et C++ d'un côté, et pour
  la théorie des langages et la compilation de l'autre, c'est donc avec un
  naturel relatif que m'est venu l'idée de ce stage.

  Le sujet de ce stage est « contribution aux développements de GNU Bison et
  Tiger », et le lieu est le Laboratoire de Recherche et de Développement de
  l'EPITA (LRDE). C'est un cadre très différent de celui de mon précédent
  stage, et j'espérai par cela pouvoir gagner en maturité quand à la question
  de savoir quel type d'entreprise ou équipe je compte rejoindre à la fin de
  ma scolarité.

  Ce stage, en plus de m'avoir donné la chance de travailler avec excellent
  mentor, m'a permis de travailler avec des technologies auxquelles je n'étais
  pas forcément familier au premier abord, et de contribuer à un projet libre.
  Cela peut paraître anodin, mais c'est en fait tout un savoir faire qu'il faut
  acquérir. J'ai donc gagné avec ce stage un certain nombre de connaissances
  techniques, mais aussi toute une méthodologie qui, pour être tout à fait
  honnête, n'était pas forcément ce par quoi je brillais le plus. J'ai
  également consolidé certaines de mes notions d'informatique fondamentale,
  notamment dans ce qui a trait à l'analyse syntaxique (plus spécifiquement
  l'analyse syntaxique LR avec anticipation) et aux automates associés.

  /* FIXME: 2 more pages. *

  \cleardoublepage
  \section{Présentation du laboratoire}
  \subsection{Présentation générale}

  Le LRDE est un laboratoire de recherche sous tutelle de l'École Pour
  l'Informatique et les Techniques Avancées (EPITA). Il est composé de 8
  chercheurs, 2 administratifs, 3 thésards et 6 étudiants-chercheurs fin 2010.
  Son financement est couvert à 90\% par l'EPITA, le reste venant de contrats
  industriels et de projets académiques.  Appartenant à une école privée, le LRDE
  est une exception dans un monde où la recherche académique scientifique est
  quasi-exclusivement du ressort d'organismes publics.

  Historique Le LRDE a été créé en 1998 à l'initiative de Joël Courtois,
  directeur de l'EPITA, qui désirait équiper l'école d’un véritable laboratoire
  académique, tant pour attirer des enseignants de qualité que pour participer à
  la reconnaissance de l'école par ses pairs. En même temps le laboratoire
  permettrait aux meilleurs élèves de l'école de s'initier au monde de la
  recherche en participant à des projets de recherche.

  En 2005, sous l'impulsion du Conseil Scientifique tout juste crée, la
  nécessité est apparue de structurer le laboratoire autour de thématiques de
  recherche afin de permettre au laboratoire de percer dans le monde académique
  et industriel. La définition de cette thématique a été difficile à trouver.
  Le point commun le plus visible était, et l'est toujours, la programmation
  générique et performante. Il s’agit clairement de la marque de fabrique du
  laboratoire que l'on retrouve dans différents projets du laboratoire et a
  donné lieu à` une dizaine de publication. Aujourd'hui le séminaire du
  laboratoire porte sur cet aspect, mais ce type de programmation est un outil
  au service de projets de recherche qui sont les réelles activités principales
  des chercheurs. Parmi ces activités le traitement d'images était déjà la
  thématique la plus importante du laboratoire mais noyée parmi les autres.
  Aussi après un long processus de réflexion, le LRDE a retenu deux thématiques
  : Reconnaissance des formes, et Automates et vérification.

  \subsection{Les projets}

  Le LRDE a trois projets phares, trois bibliothèques génériques et
  performantes écrites en C++, toutes sous licence libre. Ces bibliothèques
  font partie intégrante de notre re- cherche et peuvent être utilisées dans la
  réalisation de projets et de prestations. Actuelle- ment seule Olena, la
  bibliothèque la plus ancienne du laboratoire et la plus avancée, est utilisée
  dans le cadre de projets et a permis d’obtenir des contrats industriels.

  \subsubsection{Olena – http://olena.lrde.epita.fr/}

  Olena est une plate-forme de traitement d’images générique et performante. Le
  but de cette bibliothèque est de permettre une écriture unique d’algorithmes
  sachant que les entrées de ces algorithmes peuvent être de différente nature.
  Ainsi les entrées possibles sont des images 1D (signaux), 2D (images
  classiques), 3D (volumes), ou des graphes et leurs généralisations (complexes
  cellulaires). De plus, les valeurs stockées dans ces images sont de types
  variés : booléens pour les images binaires, des niveaux de gris avec
  différents encodages, des flottants ou autres. La force d’Olena est de
  préserver la nature abstraite des algorithmes sans pour autant devoir
  sacrifier les performances.

  \subsubsection{Vaucanson – http://vaucanson.lrde.epita.fr/}

  On peut décomposer Vaucanson en trois gros morceaux :
  \begin{enumerate}
    \item Une bibliothèque générique, dans laquelle sont définis
      \begin{enumerate}
        \item différents types de représentation d’automates (graphes ou tables
          de hachage) et d'expressions rationnelles
        \item différents types de monoïdes et semi-anneaux
        \item différents algorithmes qui les manipulent
      \end{enumerate}
    \item Une interface en ligne de commande, appelée TAF-Kit, qui permet
      d’appeler les fonctionnalités principales de la bibliothèque sans devoir
      écrire de C++. Du fait de l'approche générique employée dans la
      bibliothèque, TAF-Kit doit être instanciée pour un type particulier
      (c’est-à-dire un choix de représentation d’automate, monoïde, et semi-
      anneau). Une douzaine de telles instances sont construites pour des types
      prédéfinis.
    \item Une interface graphique dont le développement a été repris par
      l'équipe du Pr. Hsu-Chun Yen à Taïwan.
  \end{enumerate}

  Cette bibliothèque est développée selon un paradigme de programmation
  générique permettant:
  \begin{itemize}
    \item d'écrire les algorithmes une seule fois, indépendamment de la
      structure de données utilisée pour représenter les automates ou les
      expressions rationnelles, et indépendamment du monoïde et du semi-anneau
      utilisé par cet objet.
    \item de ne pas payer le prix de cette abstraction pour être (en théorie)
      aussi rapide à l'exécution qu'un algorithme
  \end{itemize}

  \subsubsection{Spot – http://spot.lip6.fr/}
  Spot est une bibliothèque de \textit{model checking}. C’est-à-dire qu'elle
  offre des algorithmes utiles à la construction d'un \textit{model checker}
  travaillant suivant une approche par automate.  Par rapport aux autres
  outils, nous nous distinguons par l'utilisation d'automates généralisés avec
  conditions d'acceptation sur les transitions, alors que la plupart de la
  recherche est faite sur des automates non-généralisés avec conditions
  d’acceptation sur les états.  Les deux formalismes sont aussi expressifs,
  mais le premier est beaucoup plus concis et permet de réaliser les opérations
  de base de l'approche automate de façon plus efficace:

  \begin{enumerate}
    \item la traduction donne des automates plus compacts, donc le produit est
      (généralement) aussi plus petits
    \item l'utilisation de condition d’acceptation généralisées, permet
      notamment de représenter très facilement des hypothèses d'équité faible
    \item la recherche de circuit acceptant dans un automate généralisé peut se
      faire aussi rapidement que dans un automate non-généralisés.
  \end{enumerate}

  \subsection{Effectifs}

  L'effectif actuel comprend les chercheurs suivants :
  \begin{adjustwidth}{-2cm}{-2cm}
    \begin{center}
      \begin{tabular}{| l | c | l | l | l |}
        \hline
        Nome & Né en & Formation & Statut & Arrivée \\
        \hline
        Ala-Eddine Ben-Salem & 1980 & M2 SLCP (INP Toulouse) & Doctorant &
        2010 \\
        Ana Stefania Calarasanu & & M2 Paris VI & Doctorante & 2012 \\
        Edwin Caralinet & 1990 & EPITA / MVA ENS Cachan & Doctorant & 2012 \\
        Etienne Renault & & M2 Paris VI & Doctorant & 2011 \\
        Yongchao Xu & 1986 & M2 Paris XI & Doctorant & 2010 \\
        \hline
        Guillaume Lazarra & 1985 & EPITA & Ing.\ de recherche & 2008 \\
        \hline
        Réda Dehak  & 1975 & Dr ENST & MdC & 2002 \\
        Akim Demaille & 1970 & X / Dr ENST & MdC & 1999 \\
        Alexandre Duret-Lutz & 1978 & EPITA / Dr Paris VI & MdC & 2007 \\
        Jonathan Fabrizio & 1978 & Dr Paris VI & MdC & 2009 \\
        Thierry Géraud & 1969 & ENST / Dr ENST & MdC & 1998 \\
        Roland Levillain & 1980 & EPITA / M2 SIRF (ENST) & MdC & 2005 \\
        Olivier Ricou & 1966 & Dr Paris VI & MdE - Directeur & 2002 \\
        Didier Verna & 1970 & ENST / Dr ENST & MdC & 2000 \\
        \hline
      \end{tabular}
    \end{center}
  \end{adjustwidth}

  \subsection{La majeure CSI}

  La majeure Calcul Scientifique et Image (CSI) est orientée vers la recherche
  académique et permet à des étudiants de s'immerger dans un laboratoire aux
  côtés des enseignants chercheurs. Les thèmes de recherche vont du traitement
  d'images à la manipulation d'automates en passant par le traitement de la
  parole, le model-checking ou l'aide à la décision.

  L'ingénieur CSI se destine dans une première phase à la préparation d'une
  thèse, en France ou à l'étranger, et il rejoindra ensuite la communauté des
  chercheurs dans un cadre académique ou au sein des structures de recherche de
  grandes entreprises ou de start-up innovantes.\footnote{%
  http://www.epita.fr/cursus-cycle-ingenieur-majeures.aspx}

  Voici le sujet du travail des sept étudiants CSI de la promotion 2013 qui
  étaient à mes côtes pendant ce stage\footnote{%
  http://lrde.epita.fr/cgi-bin/twiki/view/Publications/Seminar-2013-01-16}:

  \subsubsection*{Spot: Réduction par simulation pour les TGBA (Thomas Badie)}

  L'approche par automates du model checking s'appuie traditionnellement sur
  des Automates de Büchi (BA) qu'on souhaite les plus petits possible. Spot,
  bibliothèque de model checking, utilise principalement des TGBA qui
  généralisent les BA\@. Nous avons déjà présenté une méthode de réduction par
  simulation (dite directe). Cette technique a permis de produire des automates
  plus petits que dans les précédentes versions.  La simulation consiste à
  fusionner les états ayant le même suffixe infini. Nous montrons que nous
  pouvons aussi fusionner ceux ayant le même préfixe infini (c'est la
  cosimulation). On peut répéter la simulation et la cosimulation pour créer la
  simulation itérée. Cette méthode est incluse dans Spot 1.0 et est une grande
  amélioration de la simulation.  On expérimente aussi une méthode qui consiste
  à modifier certaines conditions d'acceptations (appellées sans importances).
  Puisque celles qui sont sur les transitions entre composantes fortement
  connexes n'ont pas d'influence sur le langage, on peut les modifier pour
  aider la simulation.

  \subsubsection*{Spot: Méthodes de réduction par ordre partiel adaptatives
  (Pierre Parutto)}

  Le model checking explicite de systèmes concurrents souffre d'une croissance
  exponentielle du nombre d'états représentant un système.  Les méthodes de
  réduction par ordre partiel sont un ensemble de méthodes permettant de
  combattre ce problème. Celles-ci permettent d'ignorer les états redondants
  lors de la génération de l'espace d'états. Parmi elles, nous avons choisi les
  algorithmes two phase et ample set comme base pour nos investigations.
  Ceux-ci ont été implémentés dans Spot, la bibliothèque C++ de model checking
  développée au LRDE, en utilisant l'interface DiVine. En se basant sur ces
  méthodes et sur le fait que les algorithmes dans Spot sont calculés à la
  volée, nous avons défini une nouvelle classe de méthodes appelées méthodes de
  réduction par ordre partiel adaptatives. L'idée est de se baser sur l'état
  courant de l'automate de la formule et non sur la formule tout entière. Les
  résultats obtenus sur notre suite de tests montrent que cette méthode donne
  de meilleurs résultats que les
  méthodes d'ordre partiel classiques.

  \subsubsection*{Speaker ID: Spherical Discriminant Analysis (Victor Lenoir)}

  Le rôle de la vérification de locuteur est de vérifier l'identité présumée
  d'un segment de parole. Actuellement, les meilleures performances sont
  obtenues par un mapping de chaque segment de parole d'un locuteur vers un
  vecteur appelé I-vector. Le score de la vérification de locuteur est calculé
  par une distance cosine entre ces deux vecteurs représentant chacun un
  locuteur. Ce rapport décrit une technique de réduction de dimension appelée
  Spherical Discriminant Analysis (SDA). Les objectifs de cette projection sont
  de maximiser la distance cosinus entre deux locuteurs différents et de
  minimiser la distance cosinus entre deux même locuteurs; il a été montré que
  le sous-espace de la SDA, qui est plus approprié pour la distance cosinus que
  la Linear Discriminant Analysis (LDA), obtient de meilleures performances en
  reconnaissance faciale. Nous allons comparer les performances obtenues par la
  SDA avec celles obtenues par la LDA.

  \subsubsection*{Climb: Parallélisation de Climb (Laurent Senta)}

  Climb est une bibliothèque de traitement d'image générique développée en
  Common Lisp. Elle fournit une couche de généricité bas niveau utilisée pour
  définir de nouveaux algorithmes de traitement d'image. Il est aussi possible
  de créer des chaînes de traitements soit en utilisant un "Domain Specific
  Language" déclaratif ou bien à l'aide d'une interface graphique. Afin
  d'améliorer les performances de la bibliothèque, ces différents niveaux
  peuvent être parallélisés. Nous décrirons les différents aspects de ce
  processus de parallélisation. Pour chaque niveau, nous détaillons
  l'implémentation des traitements parallèles, leurs impacts sur
  l'utilisabilité de la bibliothèque ainsi que les gains de performance
  obtenus.

  \subsubsection*{Vaucanson: FSMXML pour Vaucanson 2.0 (David Moreira)}

  Vaucanson est une bibliothèque de manipulation d'automates et de
  transducteurs. La version 2.0 est aujourd'hui en cours de développement et le
  design a été revu pour avoir des parties statiques et dynamiques. Dans
  Vaucanson 1.4, les entrées/sorties utilisent intensivement le format XML
  spécifié par le groupe de Vaucanson, FSMXML\@. Mes travaux consistent à
  développer et rafraîchir des spécifications du format présent dans Vaucanson
  1.4. Cette mise à jour nous permet la sauvegarde et la lecture d'automates
  aux Weight Sets particuliers tels que des expressions rationnelles ou même
  des automates pondérés.

  \subsubsection*{Olena: Calcul du flux optique dans des séquences avec des
  parties manquantes (Sylvain Lobry)}

  Calculer le flux optique peut être un premier pas vers l'inpainting vidéo.
  Pour cette application, nous devons manipuler des séquences avec des zones
  manquantes, celles à inpainter. Le flux optique peut être calculé de manière
  locale ou globale. Les méthodes globales ont généralement de meilleurs
  résultats. Dans le cas de séquences avec des zones manquantes, les méthodes
  globales ne peuvent pas être utilisées de manière directe à cause du manque
  d'informations dans ces régions. Nous présentons une méthode combinant des
  algorithmes locaux et globaux afin de calculer le flux optique dans ce type
  de séquences ce qui nous permet d'inpainter efficacement et simplement des
  vidéos.

  \subsubsection*{Olena: Variational image inpainting by combination of features
  (Coddy Levi)}

  L'inpainting consiste à réparer des parties d'une image de façon visuellement
  plausible. Une catégorie de méthodes répondant à ce problème est basée sur
  des équations aux dérivées partielles (EDP). Cette approche consiste à
  propager itérativement des informations géométriques et d'intensités à
  l'intérieur des régions à réparer. Cependant l'élaboration de tels modèles
  demande une compréhension théorique du processus de diffusion souvent
  difficile à obtenir. Basé sur le papier de Risheng Liu traitant d'une
  approche ascendante à la composition de l'EDP par combinaison d'invariants
  simples, nous proposons une mise en oeuvre optimisée que nous comparons à
  l'existant.

  \subsection{Positionnement du stage au sein du laboratoire}

  Bison est un projet GNU, ce n'est pas un projet du laboratoire. Cependant,
  son mainteneur est Akim Demaille, membre permanent du LRDE\@. De part la
  nature du projet Bison\footnote{voir section suivante}, ce stage était très
  proche de la Théorie des Langages, qui n'est pas étrangère au LRDE non plus.

  Enfin, un des projets du LRDE est le compilateur Tiger\footnote{%
  http://www.lrde.epita.fr/~akim/ccmp/tiger.html}, qui utilise Bison, et
  une partie de mon travail visait très spécifiquement des fonctionnalités
  développées avant tout pour Tiger (car non publiées).

  \cleardoublepage

  \section{Le projet GNU Bison}

GNU Bison est l'implémentation de l'analyseur syntaxique yacc par le projet GNU.

  \subsection{Analyseurs syntaxiques}

  \subsection{Présentation de bison}

  \subsection{Age et volumétrie du projet}

  \cleardoublepage

  \section{Travail effectué}

  \subsection{Vue d'ensemble}

  Mon stage a débuté septembre 2012. Peu avant moi, en juin 2012, Victor
  Santet (EPITA promotion 2015) était venu y faire un mois de stage. Il est
  parti en laissant des bases intéressantes pour une poursuite du travail dans
  sa lancée. Le travail en question portait sur les avertissements et erreurs
  générés par Bison. J'ai donc commencé mon stage en continuant sur cette
  lancée, et ce pendant un mois. Cette tâche correspond à l'élément [1] dans le
  diagramme de Gantt fourni plus bas.


  La continuation a --là encore-- été guidé très largement par les suggestions
  de mon mentor.  Il y avait déjà une liste d'améliorations possibles maintenue
  dans un fichier TODO, ce fut une source d'inspiration assez facile pour
  débuter.

  C'est ainsi que la suite de mon travail porta sur une option non essentielle
  de Bison: \texttt{-{}-graph}, qui génère une visualisation de l'automate utilisé
  par l'analyseur généré. Ce graphe était relativement pauvre, et je l'ai donc
  enrichi. Suite à diverses complications, cette tâche pris environ un mois et
  demi. Cette phase de développement se découpa en deux phases bien distinctes,
  j'ai donc pris le soin de les séparer ci-après. Le lecteur les retrouvera aux
  indices [3] et [4].

  Vinrent à ce moment plusieurs nouvelles pistes d'amélioration, lancées par
  Akim. La première ([5]) était la résolution d'un "bug" (en fait c'est n'est
  pas vraiment un, comme je l'explique plus loin dans ce rapport) de longue
  date qui faisait que si l'utilisateur ne spécifiait pas d'argument
  supplémentaire pour l'interface de la fonction de rapport d'erreur alors dans
  certains squelettes les informations de localisation de l'erreur étaient
  absentes (mais dans d'autres si). La deuxième était une amélioration de
  l'affichage des erreurs ([2]). En effet, le choix avait été fait de respecter
  des conventions semblables à celles de GCC (\textit{GNU Compiler
  Collection}), et donc l'introduction récente de changements dans le rapport
  des erreurs par celui-ci méritait une imitation dans Bison. J'ai travaillé
  sur ces deux aspects là en parallèle (bien que, contrairement à ce qu'on
  pourrait croire en retenant que ces deux fonctionnalités portent sur les
  messages d'erreurs, celles-ci étaient tout à fait orthogonales).

  Ce fut alors le premier jalon de ce stage: la sortie de Bison 2.7, qui
  incluait mes contributions de ces 3 premiers mois de stage.

  Fort des connaissances acquises, je commençai alors à travailler sur les
  squelettes de Bison.

  /* FIXME: wrapper, locations, \ldots glr.cc [8] */

  J'implémentai alors quelques modifications au squelette LALR C++ ([9]) mais
  je me suis heurté à ce qu'on pourrait appeler un mur: le schéma utilisé pour
  représenter les symboles n'était pas du tout viable si on voulait envisager
  des améliorations du genre de ce que je proposais (et qui, honnêtement,
  paraissent maintenant bien mineures par rapport à l'avalanche de travail
  engendré par ce qui a suivi). On a donc du ensuite retravailler le
  fonctionnement des symboles ([10]).

  \begin{adjustwidth}{-4cm}{-2cm}
  \begin{gantt}[xunitlength=0.7cm,fontsize=\small,titlefontsize=\small,drawledgerline=true]{14}{20}
    \begin{ganttitle}
      \titleelement{2012}{16}
      \titleelement{2013}{4}
    \end{ganttitle}
    \begin{ganttitle}
      \numtitle{9}{1}{12}{4}
      \numtitle{1}{1}{1}{4}
    \end{ganttitle}
    \begin{ganttitle}
      \numtitle{1}{7}{22}{1}
      \numtitle{1}{7}{22}{1}
      \numtitle{1}{7}{22}{1}
      \numtitle{1}{7}{22}{1}
      \numtitle{1}{7}{22}{1}
    \end{ganttitle}
    \ganttbar{warnings as errors [1]}{1}{4}
    \ganttbarcon{caret diagnostics [2]}{10}{2}
    \ganttbar{show reductions -g [3]}{5}{2}
    \ganttbarcon{show reductions xslt [4]}{7}{4}
    \ganttbar{api.pure full [5]}{10}{2}
    \ganttmilestonecon{version 2.7 released}{14}
    \ganttbar[pattern=north east lines, color=red]{epita coding-style [6]}{12}{2}
    \ganttbar{leak hunting [7]}{13}{2}
    \ganttbar{glr.cc [8]}{15}{1}
    \ganttbarcon{lalr1.cc [9]}{16}{1}
    \ganttbarcon{symbols [10]}{17}{3}
  \end{gantt}
  \end{adjustwidth}

  \subsection{Erreurs et avertissements}

  Bison est un compilateur, il va donc tenter de construire un fichier en
  sortie à partir d'un fichier en donné en entrée par l'utilisateur. Du coup,
  rien ne garantit la validité de celui-ci. Il peut y avoir des informations
  manquantes ou incompatibles dans les données lues par Bison. De fait, les
  compilateurs sont des outils de travail avec lesquels l'utilisateur interagit
  beaucoup, et en pratique le développeur (qui est plus habitué à ce que les
  choses se passent mal que à ce qu'elles se passent bien) est d'ailleurs
  souvent plus intéressé par les problèmes rencontrés par le compilateur que
  par la sortie en assembleur elle-même: la gestion des erreurs est donc un
  aspect très important des compilateurs en général, et ceux-ci incluent Bison.

  La compatibilité avec YACC (c'est à dire le comportement décrit par l'IEEE
  1003\footnote{%
  http://pubs.opengroup.org/onlinepubs/009695399/utilities/yacc.html}) n'impose
  absolument rien sur l'affichage à produire dans le terminal en cas d'erreur.
  Le choix a donc été fait de suivre le comportement de GCC, qui distingue très
  nettement plusieurs catégories d'erreurs:

  \begin{itemize}
    \item \textit{warnings}, les simples avertissements à l'utilisateur. Ils
      indiquent que le compilateur a détecté quelque chose de suspect dans le
      code de l'utilisateur, potentiellement une erreur de sa part. Le
      programme peut tout de même continuer son exécution et produire un
      résultat plausible.
    \item \textit{complaints}, ou tout simplement \textit{errors}, les erreurs
      que le compilateur rencontre sur un bout du code de l'utilisateur. Le
      programme peut continuer à traiter le reste du code de l'utilisateur,
      et aller le plus loin possible dans son exécution avant d'être bloqué
      inconditionnellement. La compilation ne produit pas de fichier en sortie.
    \item \textit{fatal errors}, les erreurs pour lesquelles ça n'aurait aucun
      sens de continuer l'exécution plus loin. Le programme s'arrête
      immédiatement.
  \end{itemize}

  \vspace{0.5cm}

  Voici la façon dont GCC les signale, par exemple:

  \begin{verbatim}
test.c: In function ‘main’:
test.c:6:3: error: expected declaration or statement at end of input
test.c:4:9: warning: unused variable ‘ar’ [-Wunused-variable]
  \end{verbatim}

  On remarque que les messages commencent par \textit{error:} ou
  \textit{warning:} selon
  qu'il s'agisse d'une erreur ou d'un simple avertissement. On note également
  que les avertissement sont divisés en de nombreuses catégories. Par exemple,
  ici, l'avertissement est attaché à la catégorie \textit{unused-variable},
  comme indiqué en fin du message.
  Voici un extrait du manuel de GCC qui montre le type de catégories de qu'il
  existe:

  \begin{verbatim}
Warning Options
       -fsyntax-only  -fmax-errors=n  -pedantic -pedantic-errors -w -Wextra
       -Wall  -Waddress -Waggregate-return  -Warray-bounds -Wno-attributes
       -Wno-builtin-macro-redefined -Wc++-compat -Wc++11-compat -Wcast-align
       -Wcast-qual -Wchar-subscripts -Wclobbered  -Wcomment -Wconversion
       -Wcoverage-mismatch  -Wno-cpp -Wno-deprecated
  \end{verbatim}

  Ces options servent à activer les avertissements associés. Il est d'usage de
  n'en activer que quelques uns par défaut et de laisser à l'utilisateur le
  soin de sélectionner quelles catégories d'avertissements il veut que le
  compilateur lui rapporte, ainsi que lesquelles il veut au contraire qu'il
  ignore.

  Le travail de Victor, sur lequel je me suis basé, avait ajouté le support de
  cette organisation des avertissements de Bison en catégories, activables
  indépendamment.

  Voici un extrait du guide d'utilisation de Bison:

  \begin{verbatim}
Operation modes:
  -h, --help                 display this help and exit
  -V, --version              output version information and exit
      --print-localedir      output directory containing locale-dependent data
      --print-datadir        output directory containing skeletons and XSLT
  -y, --yacc                 emulate POSIX Yacc
  -W, --warnings[=CATEGORY]  report the warnings falling in CATEGORY

Warning categories include:
  `midrule-values'    unset or unused midrule values
  `yacc'              incompatibilities with POSIX Yacc
  `conflicts-sr'      S/R conflicts (enabled by default)
  `conflicts-rr'      R/R conflicts (enabled by default)
  `deprecated'        obsolete constructs
  `other'             all other warnings (enabled by default)
  `all'               all the warnings
  `no-CATEGORY'       turn off warnings in CATEGORY
  `none'              turn off all the warnings
  `error[=CATEGORY]'  treat warnings as errors
  \end{verbatim}

  Et voici une démonstration de leur usage (notez la ressemblance avec le
  comportement de GCC):

  \begin{verbatim}
  $ bison -Wdeprecated
  input.yy:2.9-15: warning: deprecated directive: ‘%define variant’, use
  ‘%define api.value.type variant’ [-Wdeprecated]
  input.yy:3.4-5: warning: empty character literal [-Wother]

  $ bison -Wno-deprecated
  input.yy:3.4-5: warning: empty character literal [-Wother]

  $ bison -Wnone
  (n'affiche rien)
  \end{verbatim}

  Notez que Bison active par défaut un certain nombre de catégories, dont
  \textit{deprecated} et \textit{other}.


  \subsubsection{Avertissements de dépréciation}

  A propos de ces avertissements de dépréciation, Victor avait préparé le
  terrain pour leur utilisation, en rendant les routines de rapport d'erreurs
  génériques non seulement entre erreurs et avertissements, mais également
  entre avertissements de catégories différentes. Cependant, les avertissements
  de dépréciation n'étaient jamais émis. La dépréciation n'était alors que
  documentée, et notée en commentaire dans le code de Bison. Une de mes tâches
  fut donc, pour générer les avertissements de l'exemple précédent,  de
  transformer ainsi\footnote{commit 2062d72, Thu Oct 18 18:00:51 2012} les
  \textit{scanner} et \textit{parser} de Bison-même, car c'est à ce niveau que
  les directives déprécies sont lues par le programme, et qu'il est toujours
  intéressant de gérer les choses le plus tôt possible pour s'en débarrasser
  pour la suite

  \begin{verbatim}
diff --git a/src/parse-gram.y b/src/parse-gram.y
index f0187fb..1624dde 100644
--- a/src/parse-gram.y
+++ b/src/parse-gram.y
@@ -317,7 +317,6 @@ prologue_declaration:
 | "%expect" INT                    { expected_sr_conflicts = $2; }
 | "%expect-rr" INT                 { expected_rr_conflicts = $2; }
 | "%file-prefix" STRING            { spec_file_prefix = $2; }
-| "%file-prefix" "=" STRING        { spec_file_prefix = $3; } /* deprecated */
 | "%glr-parser"
 (..)
diff --git a/src/scan-gram.l b/src/scan-gram.l
index 8e48148..95edacc 100644
--- a/src/scan-gram.l
+++ b/src/scan-gram.l
+#define DEPRECATED(Msg)                                         \
+  do {                                                          \
+    size_t i;                                                   \
+    complain (loc, Wdeprecated,                                 \
+              _("deprecated directive: %s, use %s"),            \
+              quote (yytext), quote_n (1, Msg));                \
+    scanner_cursor.column -= mbsnwidth (Msg, strlen (Msg), 0);  \
+    for (i = strlen (Msg); i != 0; --i)                         \
+      unput (Msg[i - 1]);                                       \
+  } while (0)
+
(...)
+  /* deprecated */
+  "%default"[-_]"prec"              DEPRECATED("%default-prec");
+  "%error"[-_]"verbose"             DEPRECATED("%define parse.error verbose");
+  "%expect"[-_]"rr"                 DEPRECATED("%expect-rr");
+  "%file-prefix"{eqopt}             DEPRECATED("%file-prefix");
  \end{verbatim}

  Il y a plusieurs choses intéressantes à noter, outre le fait que maintenant
  cet avertissement soit effectivement généré.

  Notez que dans le \textit{parser}, les constructions dépréciées n'étaient pas
  traitées spécialement, ni regroupées, mais juste discrètement annotées. Par
  souci de concision, je n'ai gardé qu'une seul telle ligne ci-dessus, mais il
  y en avait trois dans ce fichier. J'ai déplacé la reconnaissance de ces
  motifs un cran plus bas, dans le \textit{scanner}-même (le plus tôt, le
  mieux c'est), où neuf autres directives dépréciées se situaient, et qui elles
  n'étaient même pas marquées comme telles. Ces directives sont maintenant
  regroupées dans un paragraphe, et utilisent tous une même macro, pour
  faciliter le travail des futurs mainteneurs.

  Dans la macro, on remarque l'appel à la fonction \texttt{complain}, par
  laquelle passe tout message d'erreur ou d'avertissement, sur laquelle j'ai
  également travaillé (détaillé plus loin).

  Également intéressant, et fruit de mon travail, la présence de \texttt{unput}
  et de la soustraction sur le curseur de position: lorsqu'une directive
  invalide dépréciée est lue, on la supprime du flux et on recommence la
  lecture en ayant inséré à la place la bonne directive (du coup le problème est
  réglé pour le reste du programme, qui peut continuer sans avoir ni à gérer
  cette directive -{}- ce qui irait à l'encontre du but de la dépréciation
  -{}-, ni à contenir du code dupliqué) mais également en ayant pris soin de ne
  pas corrompre les information de position des lexèmes, indispensables aux
  facilités de correction des bugs pour l'utilisateur, ce que aurait eu lieu si
  on avait remplacé une directive dépréciée par une autre plus longue de $n$
  caractères, Bison signalant ainsi à l'utilisateur toute erreur plus loin sur
  la même ligne comme étant décalée de $n$.

  Voici par exemple à quoi ressemblent des informations de position erronées
  par cette correction de la dépréciation:

  \begin{verbatim}
input.y:13.1-14: warning: deprecated directive, use '%define parse.error
verbose' [-Wdeprecated]
 %error_verbose %error_verbose
 ^^^^^^^^^^^^^^
input.y:13.16-29: warning: deprecated directive, use '%define parse.error
verbose' [-Wdeprecated]
 %error_verbose %error_verbose
                ^^^^^^^^^^^^^^
input.y:13.11-21: error: %define variable 'parse.error' redefined
 %error_verbose %error_verbose
           ^^^^^^^^^^^
 input.y:13-6:         previous definition
 %error_verbose %error_verbose
      ^
  \end{verbatim}

  Notez le \textit{caret diagnostic} du troisième message qui souligne quelque
  chose qui ne correspond pas à l'erreur.

  Mais ce cas est en fait bien plus vicieux. Ce bug est toujours présent à ce
  jour, d'ailleurs (alors que ceux que j'évoquais précédemment ne le sont plus,
  mais ressemblaient exactement à ceci, et j'évite en trichant ainsi avec les
  exemples de multiplier les exemples) Comme la nouvelle façon de demander de
  la verbosité est de définir la variable \texttt{parse.error}, et qu'on a
  effectué cette substitution plusieurs fois, on se retrouve avec une variable
  redéfinie.  L'erreur se situe dans le code de substitution, pas dans le code
  de l'utilisateur, notre système de localisation des erreurs perd donc
  totalement les pédales et \underline{ment}. D'ailleurs, le lecteur attentif
  aura constaté que le problème de dépréciation, qui relève du \textit{warning}
  a généré une \textit{error}, c'est très gênant car cela implique que si
  l'utilisateur avait fait le choix d'invoquer Bison avec l'option
  \texttt{-Wno-deprecated}, il n'aurait pas ces avertissements pour lui mettre
  la puce à l'oreille, il aurait juste une erreur, avec une \textit{location}
  \underline{fausse} (ce qui aurait été le cas également avec une erreur
  présente dans le code de l'utilisateur, mais au moins il aurait eu une chance
  de l'y trouver alors que là elle n'existe juste pas).

  Le lecteur vraiment attentif, lui, aura remarqué le dernier message, qui lui
  commence différemment des autres: ni par \textit{error}, ni par
  \textit{warning}. Il s'agit d'une information de contexte.

  \subsubsection{Informations de contexte, et préfixage}

  Les compilateurs fournissent souvent à l'utilisateur des informations
  supplémentaires concernant les erreurs, pour aider l'utilisateur. Voici un
  exemple avec gcc-4.7\footnote{%
  http://gcc.gnu.org/wiki/ClangDiagnosticsComparison}:

  \begin{verbatim}
  deduce.cc: In function 'void g()':
  deduce.cc:6:10: error: no matching function for call to 'f(A&)'
  deduce.cc:6:10: note: candidate is:
  deduce.cc:1:24: note: template<class T> void f(typename T::type)
  deduce.cc:1:24: note:   template argument deduction/substitution failed:
  deduce.cc: In substitution of 'template<class T> void f(typename T::type)
  [with T = A]':
  deduce.cc:6:10:   required from here
  deduce.cc:1:24: error: no type named 'type' in 'struct A'
  \end{verbatim}

  Il y a trois type de messages dans cet affichage qui ne sont pas des erreurs,
  mais qui sont simplement des indications supplémentaires.

  \begin{itemize}
    \item Le premier, \og \textit{In function `void g()'} \fg permet de
      localiser l'erreur d'une façon plus significative pour l'utilisateur. Il
      n'y a pas vraiment de situation dans Bison où l'on aurait matière à
      préciser quelque chose de genre.
    \item Le deuxième, \og \textit{deduce.cc:1:24: note: template<class T> void
      f(typename T::type)} \fg ce sont les messages qui commencent par
      \textit{note}, et qui sont des précisions complémentaires, souvent des
      indices quant à des façons de résoudre le problème. Là encore, Bison n'a
      pas grand chose de très intéressant à ajouter, les erreurs elles-mêmes
      étant généralement plutôt explicites.
    \item Enfin, la ligne qui va nous intéresser est celle qui lit \og
      \textit{deduce.cc:6:10:   required from here}\fg. Le message qui précède
      fait référence à plusieurs \textit{location}, or il est d'usage de donner
      celles-ci avec ce format très particulier \texttt{fichier:ligne:colonne}
      en début de ligne, il est donc naturel de donner les \textit{location}
      supplémentaires via des messages additionnels. Le corps de ces messages
      est indenté par rapport aux précédents.
  \end{itemize}

  Dans Bison, le cas des erreurs faisant référence à un bout du code situé en
  amont est fréquent. C'est par exemple le cas des variables (re)définies avec
  \texttt{\%define}, ou des tokens dont le numéro est en conflit, comme ici:

  \begin{verbatim}
r.y:10.10-22: error: user token number 112 redeclaration for HEX_1
r.y:9.8-16:       previous declaration for DECIMAL_1
  \end{verbatim}

  Ce second message est indenté, dans l'esprit de la remarque précédente, pour
  mettre en évidence la nature contextuelle de l'information.

  Cette indentation est un produit de mon travail. Auparavant, une routine
  existait pour afficher un message indenté: il s'agit de
  \texttt{complain\_at}.  Cette routine ne servait malheureusement que pour
  certains messages, celui ci-dessus par exemple utilisait la simple procédure
  \texttt{complain}, et n'était aucunement indentée. Voici un exemple des
  lègères corrections qui furent nécessaire à la bonne mise en forme de ces
  messages\footnote{commit cbaea01, Wed Sep 26 11:49:19 2012}:

  \begin{verbatim}
 static void
 semantic_type_redeclaration (semantic_type *s, const char *what, location first,
                              location second)
 {
-  complain_at (second, _("%s redeclaration for <%s>"), what, s->tag);
-  complain_at (first, _("previous declaration"));
+  unsigned i = 0;
+  complain_at_indent (second, &i, _("%s redeclaration for <%s>"), what, s->tag);
+  i += SUB_INDENT;
+  complain_at_indent (first, &i, _("previous declaration"));
 }
  \end{verbatim}

  Le lecteur assidu aura noté la présence d'une fonction \og \texttt{\_} \fg,
  qui était également présente dans le précédent extrait de code. Il s'agit en
  fait d'un \textit{wrapper} autour de la fonction de traduction\footnote{%
  http://translationproject.org/html/welcome.html} \texttt{gettext}, définit
  comme suit:

  \begin{verbatim}
  src/system.h: # define \_(Msgid)  gettext (Msgid)
  \end{verbatim}

  Comme les messages sont traduit selon une collection de motifs,, il faut bien
  faire attention à ne pas les découper n'importe comment, sinon on rend le
  travail des traducteurs difficile. A propos des traductions délicates, voici
  un exemple de message étrangement traduit:

  \begin{verbatim}
  $ LC_ALL=en_US.utf8 bison -Wdeprecated t.y
  t.y:5.1: error: rule given for a, which is a token
  $ LC_ALL=fr_FR.utf8 bison -Wdeprecated t.y
  t.y:5.1: erreur: la règle pour a, qui est un terminal
  \end{verbatim}

  Pour en revenir à notre exemple, on peut s'intéresser à l'implémentation de
  cette fonction \texttt{complain\_at}. Comme on peut le voir, on appelle aussi
  cette fonction pour la partie non-indentée du message. En fait, on initialise
  un entier à zéro, et on le passe à notre fonction par référence afin qu'elle
  soit modifiée. Lorsque cette valeur est nulle, la fonction se comporte comme
  la simple \texttt{complain}, au détail près qu'elle modifie cette valeur pour
  y stocker la colonne courante du début du message. Lors du second appel, la
  valeur n'est plus nulle, et \texttt{complain\_at} a un nouveau comportement:
  elle affiche le corps du message  à la colonne correspondant à la valeur
  stockée dans la variable. L'incrément manuel de cette valeur entre les deux
  appels correspond donc au changement de niveau d'indentation. Cette interface
  peu sembler un peu compliquée au premier abord: on aurait pu se contenter
  d'ajouter un Booléen indiquant le corps du message était à indenter ou non.
  Ce que l'on gagne avec cette méthode est la gestion des niveaux d'indentation
  imbriqués, au cas où un jour où en aurait l'utilité.

  Une faiblesse de cette approche (qui aurait également été présente dans la
  version plus simple que je viens de suggérer) est que l'on part du postulat
  que la partie gauche du second message (la \textit{location}) aura la même
  taille que celle du premier message. C'est généralement vrai, la différence
  potentielle se limitant généralement à quelques colonnes, et généralement
  dans le sens négatif (c'est à dire que la seconde est plus courte que la
  première, conséquence directe du fait que ce second message fait à priori
  référence à une déclaration précédente, et donc avec un numéro de ligne et/ou
  colonne inférieur --donc à potentiellement moins de chiffres), mais dans le
  cas contraire, la différence empièterait sur l'indentation. Exemple adapté du
  précédent:

  \begin{verbatim}
r.y:1.1-2: error: user token number 112 redeclaration for HEX_1
r.y:1000.18-26:previous declaration for DECIMAL_1
  \end{verbatim}

  Notez que le \textit{previous:} est à la même position relativement au
  \textit{error:}, mais que du fait du changement de taille de la
  \textit{location} l'indentation est maintenant absente.

  Ce cas ne devrait jamais se produire (la seule situation plausible étant non
  pas une différence dans les numéros de ligne mais de colonne, or on imagine
  difficilement un scénario raisonnable à mille colonnes), mais il existe une
  façon de le gérer\footnote{Déjà évoquée sur la mailing-liste:
  http://lists.gnu.org/archive/html/bison-patches/2009-09/msg00086.html}: il
  faudrait utiliser un \textit{buffer} pour stocker les messages, en y insérant
  des marqueurs là où on souhaite changer le niveau d'indentation, et ensuite
  effectuer une deuxième passe pour effectuer les insertions de blancs.

  Enfin, dernier point d'intérêt sur ce sujet: le préfixe des erreurs. En
  effet, le lecteur a très certainement constaté que nos messages d'erreurs
  commencent par \textit{error:} tandis que les avertissement, eux, commencent
  par \textit{warning:}. C'est explicite, et ça semble très naturel. Pourtant,
  ce sont des additions récentes: Victor a préfixé les avertissements, et c'est
  moi même qui ai pris la responsabilité de préfixer les erreurs.

  \subsubsection{\textit{warnings as errors}}


  Une fonctionnalité remarquable de GCC est d'activer le traitement des
  avertissements en erreurs pour certaines catégories. Par exemple, avec
  l'option \texttt{-Werror=unused-argument}, les arguments d'une fonction qui
  ne sont pas utilisés par celle-ci sont traités comme des erreurs.

  Maintenant que Bison dispose de catégories de d'avertissements, il était
  devenu aisé d'implémenter ce comportement, et ce fut en fait la première
  tâche qui m'avait été assignée (bien que j'ai été un peu distrait en cours de
  route par d'autres considérations d'ergonomie sur les erreurs, détaillées
  précédemment).

  L'option \texttt{-Werror}, permettant de promouvoir la totalité des
  avertissements en erreurs, existait déjà. En fait, c'était un hack: cette
  option active une fausse catégorie \textit{error} d'avertissements qui sert,
  dans la routine de génération des avertissements et erreurs, à promouvoir
  ceux-ci en ces dernières lorsqu'elle est activée. Ceci n'offre aucune
  flexibilité, c'est donc une impasse. Il a fallu repenser la façon de faire.

  Une grande contrainte dans l'ajout du support d'une nouvelle option dans la
  ligne de commande est que l'on veut au maximum s'intégrer dans les fonctions
  de reconnaissance d'arguments actuelles. En l'occurrence, Bison dispose de
  routines génériques et assez élégantes pour reconnaitre des consructions de
  la forme \texttt{-Wno-deprecated,other,no-yacc -{}-report=state,solved} pour
  factoriser \texttt{-Wno-deprecated, -Wother -Wno-yacc -{}-report=state
  -{}-report=solved}. On note au passage le support des variantes \texttt{no-}
  de chaque catégorie, qui se fait sans la moindre duplication de code.

  Voici un extrait de la fonction \texttt{flags\_argmatch} telle qu'elle
  existait à mon arrivée:

  \begin{verbatim}
      args = strtok (args, ",");
      while (args)
        {
          int no = strncmp (args, "no-", 3) == 0 ? 3 : 0;
          int value = XARGMATCH (option, args + no, keys, values);

          /* operations sur warn_flags ici */
  \end{verbatim}

  Selon la valeur de \texttt{no}, le traitement sur la variable globale
  \texttt{warn\_flags} stockant (sous forme de bits) les \textit{flags} activés
  est différent: il peut s'agit d'une mise d'un bit à 1, ou à 0.

  J'ai décidé de m'inspirer de cette façon de faire. J'ai introduit une
  nouvelle globale, un véritable miroir de \texttt{warn\_flags}, que j'ai nommé
  \texttt{error\_flags}, et qui stocke de la même façon l'information \og est
  ce que cet avertissement est à traiter en erreur? \fg. Du coup, l'unique
  différence entre les façons de traiter \texttt{-W[no-]category} et
  \texttt{-W[no-]error=category} est la variable à modifier: dans un cas, on
  veut modifier la reconnaissance d'une catégorie, dans l'autre cas on veut
  activer son passage en erreur.

  Le résultat se voit dans ces quelques lignes\footnote{%
  commit 20964c3, Mon Oct 1 15:01:03 2012}:

  \begin{verbatim}
    for (args = strtok (args, ","); args; args = strtok (NULL, ","))
      {
        size_t no = STRPREFIX_LIT ("no-", args) ? 3 : 0;
        size_t err = STRPREFIX_LIT ("error", args + no) ? 5 : 0;

        flag_argmatch (option, keys,
                       values, all, err ? &errors_flag : flags,
                       args, no, err);
      }
  \end{verbatim}

  Ce bout de code constitue en fait le corps de la nouvelle fonction
  \texttt{flags\_argmatch}. Les opérations sur les variables globales ont été
  déplacées dans une nouvelle fonction, pour une meilleur segmentation du code.

  Voici ces fameuses opérations, avec des explications pas à pas:

  \begin{itemize}
    \item Première partie de l'alternative (\texttt{-W[no-]category}): ces
      opérations font de simples opérations bits à bits: un masque
      (\texttt{\&=$\sim$}) dans le cas du \texttt{no-}, afin de
      désactiver le bit courant tout en préservant l'état des autres bits; un
      OU (\texttt{|=}) dans le cas normal, pour mettre le bit en question à un.


  \begin{verbatim}
    if (value)
      {
        if (no)
          *flags &= ~value;
        else
          {
  \end{verbatim}

    \item Il y a une action en plus à effectuer dans le
      cas où on fait du \texttt{error=}. En effet, par choix (cohérent avec
      celui de GCC), on veut que \texttt{-Werror=category} ne fasse pas
      qu'activer la promotion de cette catégorie en erreur, mais également
      qu'il active les \textit{warnings} de cette catégorie. Sinon,
      l'utilisateur se retrouve à devoir spécifier deux fois ce nom de
      catégorie: une fois pour le \texttt{-W}, et une fois pour le
      \texttt{-Werror=}. Ainsi, il faut travailler sur les deux variables en
      même temps.

  \begin{verbatim}
            if (err)
              warnings_flag |= value;
            *flags |= value;
          }
      }
  \end{verbatim}

    \item Deuxième partie de l'alternative (\texttt{-W[no-]none}): cette valeur
      est spéciale: plutôt que signifier la modification d'une catégorie
      précise, elle affecte au contraire toutes les autres (et éventuellement
      elle-même, cette catégorie n'ayant aucune utilité en soi pour le reste
      des évènements): c'est à dire que lorsqu'on \textit{set} \texttt{none},
      en fait ce qu'on veut c'est \textit{reset} tout le reste, d'où le code
      qui reprend le cas "usuel", mais inversé. Notons que la catégorie
      \texttt{all} est gérée encore différemment: c'est une catégorie un peu
      comme les autres, sauf que sa valeur dans l'énumération C des catégories
      (qui sont des bits: 1, 2, 4, 8, etc.) est \texttt{$\sim$all}, c'est à
      dire une valeur qui correspond "magiquement" à l'union de toutes les
      autres valeurs possibles de l'énumération.

  \begin{verbatim}
    else
      {
        if (no ? !err : err)
          *flags |= all;
        else
          *flags &= ~all;
      }
  \end{verbatim}


    \item Le mystérieux ternaire code le comportement suivant: de même que
      \texttt{-Wno-none} signifie \texttt{-Wall},
      \texttt{-Wno-error=none}\footnote{%
        Le lecteur qui a du mal à s'y retrouver peut imaginer ceci: on précise
        que, pour aucun avertissement, il ne faut pas promouvoir ceux-ci en
        erreur; c'est à dire qu'on donne une précision qui ne s'applique à
      rien, et qui n'a donc aucune importance.}, signifie
      \texttt{-Werror=all}\footnote{Il faut bien comprendre que le "error=" ne
        change pour ainsi dire rien, si ce n'est la variable courante: on peut
        donc "simplifier" ce terme de l'option, et le résultat devient
      immédiat, puisqu'on vient de rappeler que no-none est équivalent à all.}
      ce qui est faux. L'usage de \texttt{-Wno-error=category} n'activant pas
      implicitement le
      \texttt{-Werror} des autres catégories, il n'y a aucune raison pour que
      le \texttt{-Wno-error=none} le fasse.
    \item Les constructions fantaisistes à la \texttt{-Wno-error=no-category}
      sont impossibles, de par le fonctionnement du \textit{wrapper}
      \texttt{flags\_argmatch}.
  \end{itemize}


  Résultat final:


  \begin{verbatim}
  $ bison -Werror=deprecated
  input.yy:2.9-15: error: deprecated directive: ‘%define variant’, use ‘%define
  api.value.type variant’ [-Werror=deprecated]
  input.yy:3.4-5: warning: empty character literal [-Wother]
  \end{verbatim}

  \subsubsection{\textit{caret diagnostics}}

  Une caractéristique très appréciée de Clang par la communauté est sa façon de
  rapporter les erreurs, en adjoignant à chacune une ligne du code source
  fautif correspondant, en soulignant l'endroit de l'erreur. C'est ce qu'on
  appelle des \textit{caret diagnostics}, ou encore des \textit{caret errors}.

  Ils sont introduits dans gcc-4.8 également. Ca ressemble à ceci\footnote{%
  http://gcc.gnu.org/wiki/ClangDiagnosticsComparison}:

  \begin{verbatim}
$ gcc-4.8 -fsyntax-only t.c

t.c: In function 'f':
t.c:6:11: error: invalid operands to binary / (have '__m128' and 'const int *')
   myvec[1]/P;
           ^
$ clang-3.1 -fsyntax-only t.c

t.c:6:11: error: can't convert between vector values of different size ('__m128'
and 'int const *')
  myvec[1]/P;
  ~~~~~~~~^~
  \end{verbatim}

  Bison ne dispose pas actuellement d'assez d'informations pour un affichage
  aussi évolué que celui de Clang, mais nos \textit{location} sont tout à fait
  suffisantes pour imiter le comportement de GCC\@. En fait, comme l'exemple le
  montre, GCC se contente souvent de montrer la colonne de début du lieux
  intéressant, tandis que dans Bison nous avons des informations un peu plus
  riches: on garde trace de la colonne (et de la ligne) de début, mais aussi de
  fin. Par exemple, dans l'exemple suivant, on voit des \textit{location} d'une
  largeur de quatre colonnes.

  La question s'est posée de savoir comment optimiser la routine d'affichage de
  ces \textit{carets}:
  \begin{verbatim}
$ bison -fcaret input.y

input.y:10.13-17: error: %destructor redeclaration for foo
 %destructor {baz} "foo"
             ^^^^^
input.y:5.13-17:      previous declaration
 %destructor {bar} foo
             ^^^^^
\end{verbatim}

  On voit que nous avons fait le choix de souligner l'intégralité du lieux, et
  pas juste la première colonne. On aurait pu vouloir tenter une imitation de
  Clang, et utiliser \verb|^~~~|, mais comme on aurait jamais pu
  mettre l'accent sur autre chose que le premier caractère, l'intérêt n'était
  pas flagrant.


  \subsubsection{La suite de tests}

  \subsection{Visualisation graphique}
  \subsubsection{Affichage des réductions}
  \subsubsection{Adaptation à la production via XML}

  \subsection{Squelettes}
  \subsubsection{Introduction d'une nouvelle forme de pureté}
  \subsubsection{Modifications mineures de glr.cc}
  \subsubsection{Changements de lalr1.cc}
  \subsubsection{Réécriture des symboles}

  \cleardoublepage
  \section{Conclusion}

  \cleardoublepage
  \section{Annexes}

  \subsection{git shortlog --author=ranquet}
  \begin{adjustwidth}{-2cm}{-2cm}
    %\begin{verbatim}
commit 77509cece3666d88892c9cdff01cd3c4779037fc
Author: Theophile Ranquet <ranquet@lrde.epita.fr>
Date:   Tue Jan 29 14:53:35 2013 +0100

    variants: remove the 'built' assertions
    
    When using %define parse.assert, the variants come with additional variables
    that are useful for development purposes. One is a boolean indicating if the
    variant is built (to make sure we don't read a non-built variant), and the
    other is a string describing the stored type. There is no need to have both of
    these, the string is enough.
    
    * data/variant.hh (built): Remove.

\end{verbatim}
\line(1,0){250}
\begin{verbatim}
commit ee9cf8c4a6bf34fe291ed860e0d7779d78a5bcb0
Author: Theophile Ranquet <ranquet@lrde.epita.fr>
Date:   Mon Jan 28 18:26:04 2013 +0100

    m4: generate a basic_symbol constructor for each symbol type
    
    Recently, there was a slightly vicious bug hidden in the make_ functions:
    
      parser::symbol_type
      parser::make_TEXT (const ::std::string& v)
      {
        return symbol_type (token::TOK_TEXT, v);
      }
    
    The constructor for symbol_type doesn't take an ::std::string& as
    argument, but a constant variant.  However, because there is a variant
    constructor which takes an ::std::string&, this caused the implicit
    construction of a built variant.  Considering that the variant argument
    for the symbol_type constructor was cv-qualified, this temporary variant
    was never destroyed.
    
    As a temporary solution, the symbol was built in two stages:
    
      symbol_type res (token::TOK_TEXT);
      res.value.build< ::std::string&> (v);
      return res;
    
    However, the solution introduced in this patch contributes to letting
    the symbols handle themselves, by supplying them with constructors that
    take a non-variant value and build the symbol's own variant with that
    value.
    
    * data/variant.hh (b4_symbol_constructor_define_): Use the new
    constructors rather than building in a temporary symbol.
    (b4_basic_symbol_constructor_declare,
    b4_basic_symbol_constructor_define): New macros generating the
    constructors.
    * data/c++.m4 (basic_symbol): Invoke the macros here.

\end{verbatim}
\line(1,0){250}
\begin{verbatim}
commit 858666c44302b355ff0cad76b457d57659306d6f
Author: Theophile Ranquet <ranquet@lrde.epita.fr>
Date:   Mon Jan 28 18:03:58 2013 +0100

    c++: minor stylistic changes
    
    * data/c++m4: Remove useless comment lines.
    * data/variant.hh (self_type): Use this typedef instead of variant<S>.
    (b4_symbol_constructor_define_): Remove commented-out line, and stylistic
    change (avoid blank line).

\end{verbatim}
\line(1,0){250}
\begin{verbatim}
commit b20e797a712293b4c62912d3295648885bd1d9cf
Author: Theophile Ranquet <ranquet@lrde.epita.fr>
Date:   Mon Jan 28 17:41:31 2013 +0100

    c++: better inline expansion
    
    Many 'inline' keywords were in the declarations.  They rather belong in
    definitions, so move them.
    
    * data/c++.m4 (basic_symbol, by_type): Many inlines here.
    * data/lalr1.cc (yytranslate_, yy_destroy_, by_state, yypush_, yypop_): Inline
    these as well.
    (move): Move the definition outside the struct, where it belongs.

\end{verbatim}
\line(1,0){250}
\begin{verbatim}
commit 6908c2e1f7601e72c19e1f8c16eea1116e8b6708
Author: Theophile Ranquet <ranquet@lrde.epita.fr>
Date:   Tue Jan 15 17:51:45 2013 +0100

    examples: please clang
    
    * doc/bison.texi (calc++-scanner.ll): Don't output useless yyinput function.

\end{verbatim}
\line(1,0){250}
\begin{verbatim}
commit 32f4c0a1b29ed97f82ed3c41156ac6ca856a60f2
Author: Theophile Ranquet <ranquet@lrde.epita.fr>
Date:   Tue Jan 15 18:03:39 2013 +0100

    tests: better silencing of unused argument warnings
    
    input.yy:35:44: error: unused parameter 'msg' [-Werror,-Wunused-parameter]
    void yy::parser::error (std::string const& msg)
                                               ^
    
    * tests/c++.at (C++ GLR parser identifier shadowing): Don't name unused
    argument, use YYUSE instead of a direct cast to void.

\end{verbatim}
\line(1,0){250}
\begin{verbatim}
commit 492dacbc342b4b7435bed2eac9ea99909e0b0aea
Author: Theophile Ranquet <ranquet@lrde.epita.fr>
Date:   Tue Jan 15 17:54:44 2013 +0100

    bench: compatibility for Bison <= 2.7
    
    There used to be a bug in some skeletons, which caused the expansion of
    'yylval' and 'yylloc', generating these errors:
    
    input.cc:547:16: error: expected ',' or '...' before '(' token
     #define yylval (yystackp->yyval)
                    ^
    input.yy:29:39: note: in expansion of macro 'yylval'
     int yylex (yy::parser::semantic_type *yylval)
                                           ^
    
    This bug is fixed by 'skel: better aliasing of identifiers', but a workaround
    is useful when benchmarking against older versions of Bison, which are still
    affected by the bug.
    
    * etc/bench.pl.in: Rename yylval to yylvalp and yylloc to yyllocp in base
    grammar 'list'.

\end{verbatim}
\line(1,0){250}
\begin{verbatim}
commit 733fb7c593486846eca5a60d1e3ff6880ce358d7
Author: Theophile Ranquet <ranquet@lrde.epita.fr>
Date:   Tue Jan 15 13:05:21 2013 +0100

    c++: remove useless inlines
    
    * data/c++.m4 (basic_symbol): Keep 'inline' in the prototypes, but don't
    duplicate it in the implementation.
    * data/variant.hh (variant): 'inline' is not needed when the implementation is
    provided in the class definition.

\end{verbatim}
\line(1,0){250}
\begin{verbatim}
commit 403febcac54e1d61d3196b3efb80035f883cfc85
Author: Theophile Ranquet <ranquet@lrde.epita.fr>
Date:   Tue Jan 15 12:44:37 2013 +0100

    c++: m4 stylistic change
    
    * data/c++.m4 (syntax_error): Fix the indentation of 'inline'.

\end{verbatim}
\line(1,0){250}
\begin{verbatim}
commit 60607adb3c14bd0fea3eb217b1a9fccd2c308122
Author: Theophile Ranquet <ranquet@lrde.epita.fr>
Date:   Mon Jan 14 19:25:35 2013 +0100

    c++: silence warnings
    
    * data/c++.m4 (basic_symbol<Base>::operator=): Unused parameter.
    * tests/c++.at (C++ GLR parser identifier shadowing): Here too.
    -

\end{verbatim}
\line(1,0){250}
\begin{verbatim}
commit 8b4499ad042d7ca58aac4c0d6eed1e250f8546d0
Author: Theophile Ranquet <ranquet@lrde.epita.fr>
Date:   Mon Jan 14 11:02:12 2013 +0100

    news: typos
    
    * NEWS: Fix a typo, use YYSTYPE rather than semantic_type.

\end{verbatim}
\line(1,0){250}
\begin{verbatim}
commit 016426c19515897590c00a5a6cb223d72e3230b5
Author: Theophile Ranquet <ranquet@lrde.epita.fr>
Date:   Fri Jan 11 13:47:57 2013 +0100

    carets: document default activation
    
    * NEWS: Announce it.
    * doc/bison.texi: Adjust.

\end{verbatim}
\line(1,0){250}
\begin{verbatim}
commit 0242bf04acf60bef5ecde9f8f8babccf2e277c06
Author: Theophile Ranquet <ranquet@lrde.epita.fr>
Date:   Fri Dec 28 13:33:04 2012 +0100

    carets: show them in more tests
    
    * tests/input.at, tests/named-refs.at: Here.

\end{verbatim}
\line(1,0){250}
\begin{verbatim}
commit 9c4788b7ee33cf142775a827f67bb9747ba1d2ca
Author: Theophile Ranquet <ranquet@lrde.epita.fr>
Date:   Fri Dec 28 13:32:14 2012 +0100

    carets: activate by default
    
    * src/getargs.c (feature_flag): Here.
    * tests/local.at (AT_BISON_CHECK_, AT_BISON_CHECK_NO_XML): Deactivate carets
    for the testsuite, by default.
    * tests/input.at: Adjust the locations for command line definitions.

\end{verbatim}
\line(1,0){250}
\begin{verbatim}
commit 6656c9b52a306b394fb6529f735924db0f08c1a2
Author: Theophile Ranquet <ranquet@lrde.epita.fr>
Date:   Fri Jan 11 12:38:35 2013 +0100

    variants: document move and swap
    
    * data/variant.hh (swap): Doc.
    (build): Rename as...
    (move): This, more coherent naming with clearer meaning.
    * data/c++.m4 (move): Adjust.

\end{verbatim}
\line(1,0){250}
\begin{verbatim}
commit 04816a6f3203968479591ebbe1886537674fa110
Author: Theophile Ranquet <ranquet@lrde.epita.fr>
Date:   Fri Jan 11 11:41:07 2013 +0100

    c++: privatize variant blind copies
    
    * data/variant.hh (variant, operator=): Make private.
    * data/c++.m4 (operator=): New, to avoid needing a definition of that operator
    for each class member (such as a possible variant).
    * data/glr.cc, data/lalr.cc: Add the necessary include for the abort.

\end{verbatim}
\line(1,0){250}
\begin{verbatim}
commit 99d795e8f28470838cc45bd38ce3da9503fe20fb
Author: Theophile Ranquet <ranquet@lrde.epita.fr>
Date:   Fri Jan 4 12:30:01 2013 +0100

    skel: better aliasing of identifiers
    
    * data/glr.c, data/yacc.c: Avoid emitting useless defines.
    * data/glr.cc: Restore prefixes for epilogue.

\end{verbatim}
\line(1,0){250}
\begin{verbatim}
commit 0707d0c7fabeb319933aaea1bb7aa5d68101ac47
Author: Theophile Ranquet <ranquet@lrde.epita.fr>
Date:   Fri Jan 4 18:52:21 2013 +0100

    glr.cc: fatal if using api.token.ctor without variants
    
    * data/glr.cc: Here.

\end{verbatim}
\line(1,0){250}
\begin{verbatim}
commit 462b243e1d32c5b6293163d2ebb5135962b83242
Author: Theophile Ranquet <ranquet@lrde.epita.fr>
Date:   Wed Jan 9 12:46:55 2013 +0100

    skel: correctly indent switch cases
    
    * data/bison.m4 (b4_type_action_): Here.

\end{verbatim}
\line(1,0){250}
\begin{verbatim}
commit bb1f0f52264dffe36595c65cde310ce965ec1df6
Author: Theophile Ranquet <ranquet@lrde.epita.fr>
Date:   Fri Dec 21 16:48:54 2012 +0100

    variants: assert changes
    
    * data/variant.hh (swap): More asserts can't hurt. Don't perform useless swaps.
    (build): Deactivate problematic asserts, pending further investigation.
    (variant): Prohibit copy construction.

\end{verbatim}
\line(1,0){250}
\begin{verbatim}
commit e7b26e942d1466974e40a70a30d9044c4a90fd85
Author: Theophile Ranquet <ranquet@lrde.epita.fr>
Date:   Fri Dec 21 16:46:05 2012 +0100

    lalr1.cc: use a vector for the symbol stack
    
    * data/lalr1.cc: Adjust includes.
    * data/stack.hh (push, pop): Use push_back and pop_back.
    (operator []): Access vector from the end.

\end{verbatim}
\line(1,0){250}
\begin{verbatim}
commit 1dbaf37f5cd0bc0e7c6a91b8996ae974c8d2cdc1
Author: Theophile Ranquet <ranquet@lrde.epita.fr>
Date:   Fri Dec 21 16:49:48 2012 +0100

    lalr1.cc: change symbols implementation
    
    A "symbol" groups together the symbol type (INT, PLUS, etc.), its
    possible semantic value, and its optional location.  The type is
    needed to access the value, as it is stored as a variant/union.
    
    There are two kinds of symbols. "symbol_type" are "external symbols":
    they have type, value and location, and are returned by yylex.
    "stack_symbol_type" are "internal symbols", they group state number,
    value and location, and are stored in the parser stack.  The type of
    the symbol is computed from the state number.
    
    The class template symbol_base_type<Exact> factors the code common to
    stack_symbol_type and symbol_type.  It uses the Curiously Recurring
    Template pattern so that we can always (static_) downcast to the exact
    type.  symbol_base_type features value and location, and delegates the
    handling of the type to its parameter.
    
    When trying to generalize the support for variant, a significant issue
    was revealed: because stack_symbol_type and symbol_type _derive_ from
    symbol_base_type, the type/state member is defined _after_ the value
    and location.  In C++ the order of the definition of the members
    defines the order in which they are initialized, things go backward:
    the value is initialized _before_ the type.  This is wrong, since the
    type is needed to access the value.
    
    Therefore, we need another means to factor the common code, one that
    ensures the order of the members.
    
    The idea is simple: define two (base) classes that code the symbol
    type ("by_type" codes it by its type, and "by_state" by the state
    number).  Define basic_symbol<Base> as the class template that
    provides value and location support.  Make it _derive_ from its
    parameter, by_type or by_state.  Then define stack_symbol_type and
    symbol_type as basic_symbol<by_state>, basic_symbol<by_type>.  The
    name basic_symbol was chosen by similarity with basic_string and
    basic_ostream.
    
    * data/c++.m4 (symbol_base_type<Exact>): Remove, replace by...
    (basic_symbol<Base>): which derives from its parameter, one of...
    (by_state, by_type): which provide means to retrieve the actual type of
    symbol.
    (symbol_type): Is now basic_symbol<by_type>.
    (stack_symbol_type): Is now basic_symbol<by_state>.
    * data/lalr1.cc: Many adjustments.

\end{verbatim}
\line(1,0){250}
\begin{verbatim}
commit ca42755f13e176c481a537ce2bb1ec735e09915f
Author: Theophile Ranquet <ranquet@lrde.epita.fr>
Date:   Wed Dec 26 16:26:17 2012 +0100

    bench: add %b directive to use a specific Bison
    
    For example,
      $ bench.pl -v '%s lalr1.cc & %d variant & ( %b ~/old-bison/bin/bison
        | %b ~/new-bison/bin/bison )' -g list -i 10000
    
    * etc/bench.pl.in: Here.

\end{verbatim}
\line(1,0){250}
\begin{verbatim}
commit 432a008d34bdf2178f293d144066ee64977c59f9
Author: Theophile Ranquet <ranquet@lrde.epita.fr>
Date:   Thu Dec 13 11:28:11 2012 +0100

    carets: properly display when no line feed is present
    
    * src/location.c (location_caret): finish the line with one whether or not it
    is present in input. Rewrite code without getline.
    (cleanup_caret): Reset the caret_info global.
    * bootstrap.conf: No longer require getline.

\end{verbatim}
\line(1,0){250}
\begin{verbatim}
commit c1b2677ad0a0a507010fa7dde5ff07dbe4596a10
Author: Theophile Ranquet <ranquet@lrde.epita.fr>
Date:   Thu Nov 15 17:10:35 2012 +0000

    scanner: reintroduce unput for missing end tokens
    
    Unput was no longer used since a POSIX-compatiblity issue with Flex 2.5.31,
    which has been adressed in newer versions of Flex.  See this discussion:
    <http://lists.gnu.org/archive/html/bug-bison/2003-04/msg00029.html>
    
    This partially reverts commit aa4180418fff518198e1b0f2c43fec6432210dc7.
    
    * src/scan-gram.l (unexpected_end): Here.
    * tests/input.at: Adjust for new order of error reports.

\end{verbatim}
\line(1,0){250}
\begin{verbatim}
commit f039b5180581b8aed982e4c3107ceebfb82703f8
Author: Theophile Ranquet <ranquet@lrde.epita.fr>
Date:   Tue Dec 4 16:09:24 2012 +0100

    doc: fix build dependencies
    
    Suggested by Nick Bowler
    <http://lists.gnu.org/archive/html/bug-automake/2012-12/msg00001.html>
    
    * doc/local.mk: Avoid overwriting Automake's rules.

\end{verbatim}
\line(1,0){250}
\begin{verbatim}
commit e96b1b2c452d62d0fe9cef4338c882db0b7d0691
Author: Theophile Ranquet <ranquet@lrde.epita.fr>
Date:   Tue Dec 11 13:23:44 2012 +0100

    symtab: add missing initializations
    
    * src/symtab.c (semantic_type_new): Here.

\end{verbatim}
\line(1,0){250}
\begin{verbatim}
commit ae9c90ba004680b0acfa6ef7aa457c4c0cfc43d7
Author: Theophile Ranquet <ranquet@lrde.epita.fr>
Date:   Tue Dec 11 13:16:22 2012 +0100

    symtab: fix some leaks
    
    * src/symlist.c (symbol_list_free): Deep free it.
    * src/symtab.c (symbols_free, semantic_types_sorted): Free it too.
    (symbols_do, sorted): Call by address.

\end{verbatim}
\line(1,0){250}
\begin{verbatim}
commit be27db79a5ec25e1d86799f9e88532d9cc2c55f4
Author: Theophile Ranquet <ranquet@lrde.epita.fr>
Date:   Mon Dec 10 19:28:43 2012 +0100

    tests: remove use of PARSE_PARAM
    
    * tests/header.at: Here.

\end{verbatim}
\line(1,0){250}
\begin{verbatim}
commit 0906b12cd56f9777fb684b8a257a0e56c090a93c
Merge: f3ead21 d4fe9e8
Author: Theophile Ranquet <ranquet@lrde.epita.fr>
Date:   Mon Dec 10 17:01:55 2012 +0100

    Merge remote-tracking branch 'origin/maint'
    
    * origin/maint:
      news: prepare for forthcoming release
      doc: explain how mid-rule actions are translated
      error: use better locations for unused midrule values
      doc: various minor improvements and fixes
      tests: ignore more useless compiler warnings
      tests: be robust to C being compiled with a C++11 compiler
      build: beware of Clang++ not supporting POSIXLY_CORRECT
      maint: post-release administrivia
      version 2.6.90
      build: fix syntax-check error.
      cpp: simplify the Flex version checking macro
      news: improve the carets example and fix a typo
      cpp: improve the Flex version checking macro
      carets: improve the code
      maint: update news
      build: keep -Wmissing-declarations and -Wmissing-prototypes for modern GCCs
      build: drop -Wcast-qual
      gnulib: update
    
    Conflicts:
    	NEWS
    	doc/Makefile.am
    	doc/bison.texi
    	gnulib
    	src/reader.c
    	tests/actions.at
    	tests/atlocal.in
    	tests/input.at

\end{verbatim}
\line(1,0){250}
\begin{verbatim}
commit 9318e335c809a832078e8efad4054cbb456ad1da
Author: Theophile Ranquet <ranquet@lrde.epita.fr>
Date:   Fri Dec 7 12:13:38 2012 +0100

    cpp: simplify the Flex version checking macro
    
    * src/flex-scanner,h (FLEX_VERSION): Consider YY_FLEX_SUBMINOR_VERSION
    defined.

\end{verbatim}
\line(1,0){250}
\begin{verbatim}
commit fb6040f0a8cd3ddce85bac3286392a01206f5b55
Author: Theophile Ranquet <ranquet@lrde.epita.fr>
Date:   Fri Dec 7 11:57:19 2012 +0100

    news: improve the carets example and fix a typo
    
    * NEWS: Here.

\end{verbatim}
\line(1,0){250}
\begin{verbatim}
commit c49e2f153521a86f22b2631ca8a1fb6389c3a70e
Author: Theophile Ranquet <ranquet@lrde.epita.fr>
Date:   Thu Dec 6 13:21:36 2012 +0100

    cpp: improve the Flex version checking macro
    
    * src/flex-scanner.h (FLEX_VERSION): Here.

\end{verbatim}
\line(1,0){250}
\begin{verbatim}
commit dbda56040076e602f3494c5499a50d0452b162b5
Author: Theophile Ranquet <ranquet@lrde.epita.fr>
Date:   Thu Dec 6 13:17:55 2012 +0100

    carets: improve the code
    
    * src/location.c: Remove duplicate documentations.
    (caret_info): Stylistic change.
    (location_caret): Many reworks.

\end{verbatim}
\line(1,0){250}
\begin{verbatim}
commit f3ead217b8636f623399e66bd937b1c51774d4af
Merge: d6dc4d3 9960a6a
Author: Theophile Ranquet <ranquet@lrde.epita.fr>
Date:   Thu Dec 6 11:43:02 2012 +0100

    Merge remote-tracking branch 'origin/maint'
    
    * origin/maint:
      misc: pacify the Tiny C Compiler
      cpp: make the check of Flex version portable
      misc: require getline
      c++: support wide strings for file names
      doc: document carets
      tests: enhance existing tests with carets
      errors: show carets
      getargs: add support for --flags/-f
    
    Conflicts:
    	doc/bison.texi
    	m4/.gitignore
    	src/complain.c
    	src/flex-scanner.h
    	src/getargs.c
    	src/getargs.h
    	src/gram.c
    	src/main.c
    	tests/headers.at

\end{verbatim}
\line(1,0){250}
\begin{verbatim}
commit 9960a6ae75842aa7836f39b59e82eef0319338bc
Author: Theophile Ranquet <ranquet@lrde.epita.fr>
Date:   Thu Dec 6 10:49:12 2012 +0100

    misc: pacify the Tiny C Compiler
    
    * src/graphviz.c (conclude_red): Remove a useless return.

\end{verbatim}
\line(1,0){250}
\begin{verbatim}
commit b56484a5d3f983b163287f5bf0a77b80529200cf
Author: Theophile Ranquet <ranquet@lrde.epita.fr>
Date:   Wed Dec 5 18:26:36 2012 +0100

    cpp: make the check of Flex version portable
    
    This was problematic with tcc 0.9.25
    
    * src/flex-scanner.h (FLEX_VERSION_GT): Rewrite and rename as...
    (FLEX_VERSION): This.

\end{verbatim}
\line(1,0){250}
\begin{verbatim}
commit e35cd6def7a19423a5f0fc566d844f6019df111a
Author: Theophile Ranquet <ranquet@lrde.epita.fr>
Date:   Wed Dec 5 15:27:25 2012 +0100

    misc: require getline
    
    * bootstrap.conf: Here, used by src/location.c.
    * src/getargs.c (long_options): Rename --flags to --feature.

\end{verbatim}
\line(1,0){250}
\begin{verbatim}
commit 7bada5355e10f560269825cbd658caaa473573f7
Author: Theophile Ranquet <ranquet@lrde.epita.fr>
Date:   Tue Dec 4 13:12:12 2012 +0100

    doc: document carets
    
    * NEWS: Announce it.
    * doc/bison.texi (Bison Options):  Here.

\end{verbatim}
\line(1,0){250}
\begin{verbatim}
commit 505ece51710209eb4aba50166f8f7ddc57d0562b
Author: Theophile Ranquet <ranquet@lrde.epita.fr>
Date:   Mon Dec 3 11:12:07 2012 +0100

    tests: enhance existing tests with carets
    
    * tests/actions.at: Unset value.
    * tests/conflicts.at: Rule useless due to conflicts.
    * tests/input.at: Missing terminator, unexpected end of file, command line
    redefinition of variable.
    * tests/named-refs.at: Many errors.
    * tests/reduce.at: Useless nonterminals and rules.
    * tests/regression.at: Large token.

\end{verbatim}
\line(1,0){250}
\begin{verbatim}
commit 3f5d1b2c67651a9d620946de421f2e51600b885e
Author: Theophile Ranquet <ranquet@lrde.epita.fr>
Date:   Fri Nov 30 14:34:56 2012 +0100

    errors: show carets
    
    * src/locations.c (caret_info): New, persistant information useful
    for...
    (location_caret): New, print a caret.
    (cleanup_caret): Release caret_info cleanly, call it...
    * src/main.c (main): Here.
    * src/complain.c (error_message): Call location_caret here.

\end{verbatim}
\line(1,0){250}
\begin{verbatim}
commit 0db2648930e3b6c376a539aabe368aade83ee29a
Author: Theophile Ranquet <ranquet@lrde.epita.fr>
Date:   Fri Nov 30 14:33:05 2012 +0100

    getargs: add support for --flags/-f
    
    Introduce -fdiagnostics-show-caret
    
    * src/getargs.c (flag_flag): New global.
    * src/getargs.h (flag): New enum.

\end{verbatim}
\line(1,0){250}
\begin{verbatim}
commit effd30c08d3f9d936a67c6b7a9a4253915711c74
Author: Theophile Ranquet <ranquet@lrde.epita.fr>
Date:   Fri Nov 30 15:27:54 2012 +0100

    getargs: don't label --language/-l as experimental
    
    * NEWS: Announce it.
    * doc/bison.texi, src/getargs.c (usage): Here.

\end{verbatim}
\line(1,0){250}
\begin{verbatim}
commit a37131cc63607b8fa59ba58296dd0b1682ec54d5
Author: Theophile Ranquet <ranquet@lrde.epita.fr>
Date:   Mon Mar 12 10:01:09 2012 +0100

    getargs: fix the locations of command-line input
    
    * src/getargs.c (command_line_location): Here.
    * tests/input.at: Adjust.

\end{verbatim}
\line(1,0){250}
\begin{verbatim}
commit d4e985d587a0111c8ad39cd56b815a7f63540116
Author: Theophile Ranquet <ranquet@lrde.epita.fr>
Date:   Thu Nov 15 12:02:40 2012 +0000

    errors: indent missing action code semicolon warning
    
    Also, remove a duplicate #define.
    
    * src/scan-code.l (SC_RULE_ACTION): Here.
    * tests/actions.at: Adjust.

\end{verbatim}
\line(1,0){250}
\begin{verbatim}
commit 1f1bd5729743e155e71767810eaef22bd84e340b
Author: Theophile Ranquet <ranquet@lrde.epita.fr>
Date:   Mon Nov 26 18:17:15 2012 +0100

    doc: introduce api.pure full, rearrange some examples
    
    * NEWS: Add entry.
    * doc/bison.texi (%define Summary): Show the old Yacc behaviour.
    (Parser Function): Move parse-param examples here.
    (Pure Calling): Remove parse-param examples.
    (Error Reporting): Don't show the old behavior, stick to 'full'.

\end{verbatim}
\line(1,0){250}
\begin{verbatim}
commit 6428a8a4a569b0b7ac1b84bdd78efc3fe18160ca
Author: Theophile Ranquet <ranquet@lrde.epita.fr>
Date:   Fri Nov 23 17:30:47 2012 +0000

    yacc.c: support "%define api.pure full"
    
    This makes the interface for yyerror() pure without the need for a spurious
    parse_param.
    
    * data/yacc.c (b4_pure_if, b4_pure_flag): New definition, accept three states.
    (b4_yacc_pure_if): Rename as...
    (b4_yyerror_arg_loc_if): This, and use b4_pure_flag.
    * tests/actions.at (%define api.pure): Modernize.
    * test/calc.at (Simple LALR Calculator): Modernize.
    * tests/local.at (AT_YYERROR_ARG_LOC_IF): Adjust.

\end{verbatim}
\line(1,0){250}
\begin{verbatim}
commit dbb998364f6a3057ab65a5ed4b6190396cf5eaf2
Author: Theophile Ranquet <ranquet@lrde.epita.fr>
Date:   Fri Nov 23 16:23:27 2012 +0000

    local.at: improvements
    
    * tests/local.at (AT_YYERROR_FORMALS): Make llocp const.
    (AT_PURE_AND_LOC_IF, AT_GLR_OR_PARAM_IF): Remove, expand...
    (AT_YYERROR_ARG_LOC_IF): Here, and use m4_join for readability.

\end{verbatim}
\line(1,0){250}
\begin{verbatim}
commit fb4c8a7cb97844f6c66921a77e79311a19d12fc2
Author: Theophile Ranquet <ranquet@lrde.epita.fr>
Date:   Mon Nov 19 10:43:56 2012 +0000

    yacc.c: always initialize yylloc
    
    The initial location might be used if the parser starts by an empty
    reduction, so really ensure proper initialization of the initial
    location.  The previous approach fails for PostgreSQL, which uses
    Reported by Peter Eisentraut.
    http://lists.gnu.org/archive/html/bug-bison/2012-11/msg00023.html
    With help from Théophile Ranquet.
    
    * data/yacc.c (b4_declare_scanner_communication_variables): Be sure
    to initialize yylloc, even when its structure is unknown.
    (yyparse): Simplify the call to b4_dollar_pushdef.
    * tests/actions.at (Initial location): Check of similar pattern
    as in the case of PostgreSQL.

\end{verbatim}
\line(1,0){250}
\begin{verbatim}
commit 05c93b7d844e59ecaa5dec3bd6d9091f5aa5d1b0
Author: Theophile Ranquet <ranquet@lrde.epita.fr>
Date:   Mon Nov 12 15:41:52 2012 +0000

    tests: close files in glr-regression
    
    * tests/glr-regression.at: Here.

\end{verbatim}
\line(1,0){250}
\begin{verbatim}
commit be3517b007a833ffec1735691f256f6a37e4a67f
Author: Theophile Ranquet <ranquet@lrde.epita.fr>
Date:   Tue Oct 23 15:43:54 2012 +0000

    xml: match DOT output and xml2dot.xsl processing
    
    Make the DOT produced by XSLT processing equivalent to the one made with the
    --graph option.
    
    * data/xslt/xml2dot.xsl: Stylistic changes, and add support for reductions.
    * doc/bison.texi (Xml): Update.
    * src/graphviz.c (conclude_red): Minor stylistic changes to DOT internals.
    (output_red): Swap enabled and disabled reductions output, for coherence
    with XSLT output.
    * src/print_graph.c (print_core): Minor stylistic change to States' output.
    (print_actions): Swap order of output for reductions and transitions.
    * tests/local.at (AT_BISON_CHECK_XML): Ignore differences in order.
    * tests/output.at: Adjust to changes in DOT internals.

\end{verbatim}
\line(1,0){250}
\begin{verbatim}
commit 489b320d4a70692313a059b10786913a1310f617
Author: Theophile Ranquet <ranquet@lrde.epita.fr>
Date:   Wed Nov 7 09:54:06 2012 +0000

    xml: factor xslt space template
    
    * data/xslt/bison.xsl (space): New, import from...
    * data/xslt/xml2text.xsl: Here.

\end{verbatim}
\line(1,0){250}
\begin{verbatim}
commit ccda5c9eacfd97f0064c8d6102d3c5f06cb541c1
Author: Theophile Ranquet <ranquet@lrde.epita.fr>
Date:   Fri Nov 9 16:40:45 2012 +0000

    graph: fix a memory leak
    
    * src/graphviz.c (output_red): Here.

\end{verbatim}
\line(1,0){250}
\begin{verbatim}
commit 9c16d39944f2083dde29e364c1f61aebae6f2643
Author: Theophile Ranquet <ranquet@lrde.epita.fr>
Date:   Mon Oct 22 17:37:57 2012 +0000

    xml: documentation
    
    The XML output combined with the XSL Transformations provided in data/ are
    incredibly useful, they should be documented.
    
    * doc/bison.texi (Xml): New node.

\end{verbatim}
\line(1,0){250}
\begin{verbatim}
commit d42fe46ec330a0b202b28c2af4c8171a627ab531
Author: Theophile Ranquet <ranquet@lrde.epita.fr>
Date:   Thu Oct 18 12:02:05 2012 +0000

    output: capitalize State
    
    * src/print.c (print_state): Here.
    * tests/conflicts.at, tests/existing.at, tests/local.at, tests/reduce.at,
    tests/regression.at, tests/sets.at: Adjust.

\end{verbatim}
\line(1,0){250}
\begin{verbatim}
commit 32288c8cbd309b29dff45e81d78374a5eb9c006e
Author: Theophile Ranquet <ranquet@lrde.epita.fr>
Date:   Fri Oct 26 18:07:08 2012 +0000

    graphs: fix spacing refactoring
    
    * src/print_graph.c (print_lhs, print_core): Here.

\end{verbatim}
\line(1,0){250}
\begin{verbatim}
commit ed91d427fe80ea2c5991c8862b0145613b9a0650
Author: Theophile Ranquet <ranquet@lrde.epita.fr>
Date:   Fri Oct 26 17:54:56 2012 +0000

    tests: make deprecation tests more specific
    
    * tests/input.at (Deprecated directives): Here, don't generate unrelated errors
    or warnings.

\end{verbatim}
\line(1,0){250}
\begin{verbatim}
commit e89d8806ea418f44c72aed97910ce609ad80f849
Author: Theophile Ranquet <ranquet@lrde.epita.fr>
Date:   Fri Oct 26 16:11:08 2012 +0000

    tests: fix AT_BISON_CHECK_WARNINGS_ stderr rewriting
    
    * tests/input.at (Deprecated directives): Avoid spurious error.
    * tests/locat.at (AT_BISON_CHECK_WARNINGS): Adjust for recent changes.

\end{verbatim}
\line(1,0){250}
\begin{verbatim}
commit 23ac665d2181daaa21a0d151be3f6f2a2380e014
Author: Theophile Ranquet <ranquet@lrde.epita.fr>
Date:   Fri Oct 26 18:13:44 2012 +0000

    scan-skel.l: consider m4 notes as related to "complaint" errors
    
    * src/scan-skel.l (flag): Here.

\end{verbatim}
\line(1,0){250}
\begin{verbatim}
commit ac0eca3ccbf4dada77577cc18d28857eb77087aa
Author: Theophile Ranquet <ranquet@lrde.epita.fr>
Date:   Fri Oct 26 18:13:27 2012 +0000

    warnings: distinguish context information based on warning type
    
    * src/scan-code.l (show_sub_message, show_sub_messages): Take a new warnings
    argument.

\end{verbatim}
\line(1,0){250}
\begin{verbatim}
commit 697a8022c656f8081d71d470e7b538f699af372c
Author: Theophile Ranquet <ranquet@lrde.epita.fr>
Date:   Fri Oct 26 18:12:53 2012 +0000

    warnings: fix early exit of warnings treated as errors
    
    Treating warnings as errors caused Bison to exit earlier than needed, making it
    hide warnings that would have been printed had -Werror not been set.
    
    Also, fix a bug that caused some context information of errors to not be
    shown.
    
    * src/complain.c (complaint_issued): Rename as...
    (complaint_status): This, and change its type from boolean to
    * src/complain.h (err_status): This, new enumeration.
    * src/main.c (main): Adjust (only finish early if an actual complaint was
    risen, not a mere warning treated an error).
    * src/reader.c: Adjust.

\end{verbatim}
\line(1,0){250}
\begin{verbatim}
commit 8f6bbe0c106114eec8988eadaa25d59969439986
Author: Theophile Ranquet <ranquet@lrde.epita.fr>
Date:   Thu Oct 25 12:12:28 2012 +0000

    tests: reindent for legibility
    
    * tests/local.at (AT_BISON_CHECK_WARNINGS_): Here.

\end{verbatim}
\line(1,0){250}
\begin{verbatim}
commit 6b1cbda1b9e0e8f29c0a6e4e44ce2d5a2aee974f
Author: Theophile Ranquet <ranquet@lrde.epita.fr>
Date:   Mon Oct 22 14:19:31 2012 +0000

    misc: document TESTSUITEFLAGS in README-hacking
    
    * README-hacking: Document -j and -k flags.

\end{verbatim}
\line(1,0){250}
\begin{verbatim}
commit 0f92546f47f5bdac1025ca4b423c1ebfc90dc1fa
Author: Theophile Ranquet <ranquet@lrde.epita.fr>
Date:   Mon Oct 22 11:10:53 2012 +0000

    deprecation: add tests
    
    * tests/input.at (Deprecated directives warn, Non-deprecated
    directives don't, Unput doesn't mess up locations): New tests.

\end{verbatim}
\line(1,0){250}
\begin{verbatim}
commit 25b27513d94852f670983b2165e4e2d22eeaff45
Author: Theophile Ranquet <ranquet@lrde.epita.fr>
Date:   Wed Oct 24 10:46:03 2012 +0000

    regen

\end{verbatim}
\line(1,0){250}
\begin{verbatim}
commit 2062d72deb37782eb2c842e8dd1e606db5accbd1
Author: Theophile Ranquet <ranquet@lrde.epita.fr>
Date:   Thu Oct 18 18:00:51 2012 +0000

    deprecation: issue warnings in scanner
    
    * src/parse-gram.y: Move the handling of (three) deprecated constructs ...
    * src/scan-gram.l: ...Here, and issue warnings.
    (DEPRECATED): New.

\end{verbatim}
\line(1,0){250}
\begin{verbatim}
commit 90b1335aed1edcbead0b50446b85c9a32d7c295f
Author: Theophile Ranquet <theophile.ranquet@gmail.com>
Date:   Wed Oct 3 15:26:56 2012 +0000

    maint: fix an erroneous include
    
    This fixes test 130 (Several parsers).
    
    * data/location.cc: Include <iostream> rather than <iosfwd> since
    we really need << on strings for instance.
    * NEWS: Document this.
    
    Signed-off-by: Akim Demaille <akim@lrde.epita.fr>

\end{verbatim}
\line(1,0){250}
\begin{verbatim}
commit fc4fdd623e7613c002f7c7d6cb73b4ab4bb5b494
Author: Theophile Ranquet <theophile.ranquet@gmail.com>
Date:   Thu Oct 18 15:38:32 2012 +0000

    graphs: documentation
    
    Note that 'make web-manual' fails.
    
    * NEWS: Document these changes.
    * doc/Makefile.am: Adjust to generate example files.
    * doc/bison.texi: Add a Graphviz section after "Understanding::", the section
    describing the .output file, because these are similar.
    * doc/figs/example-reduce.dot, doc/figs/example-reduce.txt,
    doc/figs/example-shift.dot, doc/figs/example-shift.txt: New, minimal
    examples to illustrate the documentation.
    
    Signed-off-by: Akim Demaille <akim@lrde.epita.fr>

\end{verbatim}
\line(1,0){250}
\begin{verbatim}
commit dd47b5220cb0346e3dd8d873f09b606733adf836
Author: Theophile Ranquet <theophile.ranquet@gmail.com>
Date:   Thu Oct 18 15:38:33 2012 +0000

    graphs: add tests, introducing -k graph
    
    * tests/output.at (AT_TEST): New.
    Use it to add 6 --graph tests.
    
    Signed-off-by: Akim Demaille <akim@lrde.epita.fr>

\end{verbatim}
\line(1,0){250}
\begin{verbatim}
commit ce6cf10f73b2eabdb460ade6e4813254459f442a
Author: Theophile Ranquet <theophile.ranquet@gmail.com>
Date:   Thu Oct 18 15:38:31 2012 +0000

    graphs: change the output format of the rules
    
    Use something similar to the report file.
    
    * src/print_graph.c (print_lhs): New, obstack equivalent of rule_lhs_print.
    (print_core): Use here.
    
    Signed-off-by: Akim Demaille <akim@lrde.epita.fr>

\end{verbatim}
\line(1,0){250}
\begin{verbatim}
commit 8048226f50f62fb64044625bf0ebbdc403a6776e
Author: Theophile Ranquet <theophile.ranquet@gmail.com>
Date:   Thu Oct 18 15:38:30 2012 +0000

    graphs: style changes
    
    * src/graphviz.c (start_graph): Use courier font.
    (conclude_red): Use commas to separate attributes. Show the acceptation
    as a special reduction, with a blue color and an "Acc" label. Show the
    lookahead tokens between square brackets.
    (output_red): No longer label default reductions.
    * src/print_graph.c (print_core): Refactor spacing, and print an
    additional space between a rule's rhs and its lookahead tokens. Also,
    capitalize "State".
    (print_actions): Style, move a declaration.
    
    Signed-off-by: Akim Demaille <akim@lrde.epita.fr>

\end{verbatim}
\line(1,0){250}
\begin{verbatim}
commit 85935600ad544bfe16ede35621f0b96b5d3dce81
Author: Theophile Ranquet <theophile.ranquet@gmail.com>
Date:   Thu Oct 18 15:38:29 2012 +0000

    graphs: address an issue with R/R conflicts
    
    All disabled reductions should now be shown as such.
    
    * src/graphviz.c (output_red): Here.
    (conclude_red): New.
    
    Signed-off-by: Akim Demaille <akim@lrde.epita.fr>

\end{verbatim}
\line(1,0){250}
\begin{verbatim}
commit f60321dc590673828a6b6aed652523927d11adc8
Author: Theophile Ranquet <theophile.ranquet@gmail.com>
Date:   Mon Oct 15 17:03:17 2012 +0000

    scan-skel.l: shift complain_args arguments
    
    Because argv[0] is never used, shift it out from the argument list.
    
    * src/complain.c (complain_args): Here.
    * src/scan-skel.l (at_complain): Adjust argv and argc.
    
    Signed-off-by: Akim Demaille <akim@lrde.epita.fr>

\end{verbatim}
\line(1,0){250}
\begin{verbatim}
commit 56f0d1d187733e2b04c68591f4d6f66443c7dd37
Author: Theophile Ranquet <theophile.ranquet@gmail.com>
Date:   Mon Oct 15 17:03:16 2012 +0000

    scan-skel.l: formatting changes
    
    * src/scan-skel.l (fail_for_at_directive_too_few_args): Here.
    
    Signed-off-by: Akim Demaille <akim@lrde.epita.fr>

\end{verbatim}
\line(1,0){250}
\begin{verbatim}
commit d79683fa95d91c3afd4de8ee010d0d013a84a222
Author: Theophile Ranquet <ranquet@lrde.epita.fr>
Date:   Thu Oct 11 16:09:03 2012 +0000

    graphs: minor style changes
    
    * src/graphviz.c (output_red): Fix C90 issues.
    Reduce variable scopes.
    
    Signed-off-by: Akim Demaille <akim@lrde.epita.fr>

\end{verbatim}
\line(1,0){250}
\begin{verbatim}
commit 83bae26d3f9b374fb189703aa8795ecb3240bab2
Author: Theophile Ranquet <ranquet@lrde.epita.fr>
Date:   Mon Oct 8 06:11:41 2012 +0000

    graphs: show reductions
    
    * src/graphviz.c (output_red): New, show reductions on the graph.
    (no_reduce_bitset_init): New, initialize a bitset.
    (print_token): New, print a lookahead token.
    (escape): New, print "foo" as \"foo\" because Dot doesn't like quotes within
    a label.
    
    * src/graphviz.h : Adjust.
    * src/print_graph.c (print_actions): Call output_red here.
    
    Signed-off-by: Akim Demaille <akim@lrde.epita.fr>

\end{verbatim}
\line(1,0){250}
\begin{verbatim}
commit 9fc99ca35009c1be76ce1c73133833ad82562a8b
Author: Theophile Ranquet <theophile.ranquet@gmail.com>
Date:   Wed Oct 10 17:14:04 2012 +0000

    graphs: style: prefix state number with "state"
    
    * src/print_graph.c (print_core): Here.
    
    Signed-off-by: Akim Demaille <akim@lrde.epita.fr>

\end{verbatim}
\line(1,0){250}
\begin{verbatim}
commit a13121f75994966dfbb7bed4de31dd7bb2516350
Author: Theophile Ranquet <ranquet@lrde.epita.fr>
Date:   Mon Oct 8 15:53:44 2012 +0000

    graphs: style: use left justification for states
    
    The label text of nodes is centered "by default" (by the use of '\n' as
    a line feed). This gives bad readability to the grammar rules shown in
    state nodes, a left justification is much nicer. This is done by using '\l'
    as the line feed.
    
    In order to allow \l in the DOT file, changes to the quoting system seem
    necessary.
    
    * src/print_graph.c (print_core): Escape tokens here, instead of...
    * src/graphviz.c (output_node): Here...
    (escape): Using this, new.
    
    Signed-off-by: Akim Demaille <akim@lrde.epita.fr>

\end{verbatim}
\line(1,0){250}
\begin{verbatim}
commit 2be37f19fe2165fcfd8752a30b55308a5bdb9b84
Author: Theophile Ranquet <theophile.ranquet@gmail.com>
Date:   Wed Oct 10 17:14:02 2012 +0000

    graphs: style: prefix rules and change shapes
    
    * src/graphviz.c (start_graph): Use box rather than ellipsis.
    * src/print_graph.c (print_core): Prefix rules with their number.
    
    Signed-off-by: Akim Demaille <akim@lrde.epita.fr>

\end{verbatim}
\line(1,0){250}
\begin{verbatim}
commit 47a31596c62b2d25dd68ca11df30dccef4944e13
Author: Theophile Ranquet <theophile.ranquet@gmail.com>
Date:   Wed Oct 10 17:14:01 2012 +0000

    obstack: import obstack_finish0 from master
    
    * src/system.h (obstack_finish0): New.
    
    Signed-off-by: Akim Demaille <akim@lrde.epita.fr>

\end{verbatim}
\line(1,0){250}
\begin{verbatim}
commit 92f39ba6dc740ac1f9597e256014ae8740b0477d
Author: Theophile Ranquet <theophile.ranquet@gmail.com>
Date:   Thu Oct 4 15:23:40 2012 +0200

    scan-skel: recognize the @directives directly in scanner
    
    * src/scan-skel.l (at_directive, at_init): New.
    (at_ptr): New, function pointer used to call the right at_directive
    function (at_basename, etc.).
    (outname): Rename as...
    (out_name): this, for consistency with out_lineno.
    
    Signed-off-by: Akim Demaille <akim@lrde.epita.fr>

\end{verbatim}
\line(1,0){250}
\begin{verbatim}
commit 896517cdf0bac6da7fabc52509205724eb1a9e28
Author: Theophile Ranquet <theophile.ranquet@gmail.com>
Date:   Thu Oct 4 15:12:56 2012 +0200

    scan-skel: split @directive functions
    
    * src/scan-skel.l (at_directive_perform): Split as...
    (at_basename, at_complain, at_output): these.
    
    Signed-off-by: Akim Demaille <akim@lrde.epita.fr>

\end{verbatim}
\line(1,0){250}
\begin{verbatim}
commit c6c8de1609da38a4ffb6dbed8047491d85d57e3d
Author: Theophile Ranquet <theophile.ranquet@gmail.com>
Date:   Thu Oct 4 10:35:42 2012 +0000

    errors: support indented context info in m4 macros
    
    * TODO: Address the issue, so remove it.
    * data/bison.m4: Use b4_error with [[note]] rather than a complain_at
    for context information.
    * src/complain.c (complain_args): Take an additional argument, an
    indentation pointer, to allow the dispatching of context information.
    * src/complain.h (complain_args): Adjust prototype.
    * src/scan-skel.l (at_directive_perform): Recognize the new @note mark.
    * tests/input.at: Adjust.
    
    Signed-off-by: Akim Demaille <akim@lrde.epita.fr>

\end{verbatim}
\line(1,0){250}
\begin{verbatim}
commit 0505df0cbae4065b17f3fb6953c68f87217e7ea2
Author: Theophile Ranquet <theophile.ranquet@gmail.com>
Date:   Thu Oct 4 10:35:41 2012 +0000

    errors: factor b4_error @directives
    
    Instead of @complain, @warn, and @fatal, use a unique @complain
    directive. This directive's first argument is "complain", "warn", etc.
    
    * data/bison.m4 (m4_error): Here.
    * src/scan-skel.l (at_directive_perform): Adjust.
    (flag): Replace the switch by safer and more explicit if branches.
    
    Signed-off-by: Akim Demaille <akim@lrde.epita.fr>

\end{verbatim}
\line(1,0){250}
\begin{verbatim}
commit b999409e09db9adfbc7a80569c4e7a8cb0587adf
Author: Theophile Ranquet <theophile.ranquet@gmail.com>
Date:   Thu Oct 4 10:35:40 2012 +0000

    errors: pointerize complain_at_indent
    
    * src/complain.c (complain_at_indent): Rename as...
    (complaint_indent): This, and take the location as a pointer.
    * src/complain.h, src/muscle-tab.c, src/reader.c, src/scan-code.l,
    src/symtab.c: Adjust.
    
    Signed-off-by: Akim Demaille <akim@lrde.epita.fr>

\end{verbatim}
\line(1,0){250}
\begin{verbatim}
commit a2b3f10183008612c4350e072d5611e0b6a8a462
Author: Theophile Ranquet <theophile.ranquet@gmail.com>
Date:   Wed Oct 3 15:26:56 2012 +0000

    maint: fix an erroneous include
    
    This fixes test 130 (Several parsers).
    
    * data/location.cc: Include <iostream> rather than <iosfwd> since
    we really need << on strings for instance.
    * NEWS: Document this.
    
    Signed-off-by: Akim Demaille <akim@lrde.epita.fr>

\end{verbatim}
\line(1,0){250}
\begin{verbatim}
commit 20964c33f91dc49b9f1b97a85505dd69776fe41c
Author: Theophile Ranquet <theophile.ranquet@gmail.com>
Date:   Mon Oct 1 15:01:03 2012 +0000

    warnings: separate flags_argmatch
    
    This function is now a mere iterator that calls flag_argmatch,
    a new function, that matches a single option parameter.
    
    * src/getargs.c (flag_argmatch): New, taken from...
    (flags_argmatch): Here.
    
    Signed-off-by: Akim Demaille <akim@lrde.epita.fr>

\end{verbatim}
\line(1,0){250}
\begin{verbatim}
commit a686f6cdc0077cea9f705f3c87bb8fc4fa27e1b4
Author: Theophile Ranquet <theophile.ranquet@gmail.com>
Date:   Mon Oct 1 15:01:02 2012 +0000

    warnings: refactoring
    
    The code here was too confusing, this seems more natural.
    
    * src/complain.c (error_message): Move the indentation check and the category
    output to complains. Also, no longer take a 'warnings' argument.
    (complains): Factor calls to error_message.
    
    Signed-off-by: Akim Demaille <akim@lrde.epita.fr>

\end{verbatim}
\line(1,0){250}
\begin{verbatim}
commit 46b7d74cb5d2f4ccefdf92f46d25cd8761324f18
Author: Theophile Ranquet <theophile.ranquet@gmail.com>
Date:   Mon Oct 1 15:01:01 2012 +0000

    formatting changes
    
    * src/complain.c: Here.
    
    Signed-off-by: Akim Demaille <akim@lrde.epita.fr>

\end{verbatim}
\line(1,0){250}
\begin{verbatim}
commit 782e8187185a009ce698ad294c87e49a0a30c3a1
Author: Theophile Ranquet <theophile.ranquet@gmail.com>
Date:   Mon Oct 1 15:01:00 2012 +0000

    warnings: organize variadic complaints call
    
    Move the dispatch of variadic complains to complain.c, rather than do
    it in a scanner.
    
    * src/complain.h, src/complain.c (complain_args): New.
    * src/scan-skel.l (at_directive_perform): Use it.
    
    Signed-off-by: Akim Demaille <akim@lrde.epita.fr>

\end{verbatim}
\line(1,0){250}
\begin{verbatim}
commit bb8e56ff67d289d15f1a4dc7f5e1502cd3bbfc24
Author: Theophile Ranquet <theophile.ranquet@gmail.com>
Date:   Mon Oct 1 13:44:20 2012 +0000

    warnings: fusion of complain and complain_at
    
    These functions are very similar, and keeping them seperate makes
    future improvements difficult, so merge them.
    
    This impacts 89 calls.
    
    * src/bootstrap.conf: Adjust.
    * src/complain.c (complain, complain_at): Merge into...
    (complain): this.
    (complain_args): Adjust.
    * src/complain.h, src/conflicts.c, src/files.c, src/getargs.c,
    * src/gram.c, src/location.c, src/muscle-tab.c, src/parse-gram.y,
    * src/reader.c, src/reduce.c, src/scan-code.l, src/scan-gram.l,
    * src/scan-skel.l, src/symlist.c, src/symtab.c:
    Adjust.
    
    Signed-off-by: Akim Demaille <akim@lrde.epita.fr>

\end{verbatim}
\line(1,0){250}
\begin{verbatim}
commit a49f4904c36d585fa057b681c38af4d48de7c23b
Author: Theophile Ranquet <theophile.ranquet@gmail.com>
Date:   Mon Oct 1 13:44:19 2012 +0000

    warnings: remove spurious suffixes on context
    
    Rectify a bug that introduced suffixes out of place.
    
    * src/complainc.c (complains): Handle all three special warning bits.
    * src/scan-code.l (show_sub_message): Remove useless argument.
    * tests/named-refs.at: Adjust.
    
    Signed-off-by: Akim Demaille <akim@lrde.epita.fr>

\end{verbatim}
\line(1,0){250}
\begin{verbatim}
commit b506d9bfcb9e96d07264a43da7bdf136329dbc86
Author: Theophile Ranquet <theophile.ranquet@gmail.com>
Date:   Fri Sep 28 12:12:59 2012 +0000

    errors: indent "user token number redeclaration" context
    
    This is the continuation of the work on the readability of errors
    context.
    
    * src/symtab.c (user_token_number_redeclaration): Use
    complain_at_indent to output with increased indentation level.
    * tests/input:at: Apply this change.
    
    Signed-off-by: Akim Demaille <akim@lrde.epita.fr>

\end{verbatim}
\line(1,0){250}
\begin{verbatim}
commit 46bdb8ec593ea0cfd1604b7d1253a3af1c571e6d
Author: Theophile Ranquet <theophile.ranquet@gmail.com>
Date:   Thu Sep 27 10:52:47 2012 +0000

    errors: don't display "warnings treated as errors"
    
    This line doesn't add any meaningful information anymore, the appended
    [-Werror=CATEGORY] is enough.  It is actually more insightful, as it
    allows to distinguish warnings treated as errors from those that
    aren't.  This line is also removed by gcc 4.8.
    
    * src/complain.c (set_warnings_issued): The only action left was
    checking if the error bit corresponding to the warning issued was set,
    and that function was only called once. Therefore, remove it, and do
    its job directly in the caller...
    (complains): here.
    * src/complains.h: Adjust.
    * tests/input.at: Adjust.
    * NEWS: Document this change.
    
    Signed-off-by: Akim Demaille <akim@lrde.epita.fr>

\end{verbatim}
\line(1,0){250}
\begin{verbatim}
commit 9503b0a4a8928afa0fdab88672f6127ee7e87963
Author: Theophile Ranquet <theophile.ranquet@gmail.com>
Date:   Thu Sep 27 10:52:45 2012 +0000

    errors: introduce the -Werror=CATEGORY option
    
    This new option is a lot more flexible than the previous one. Its
    details will be discussed in the NEWS and info file, in a forthcoming
    change.
    
    If no category is specified (ie: used as simply "-Werror"), the
    functionality is the same as before.
    
    * src/complain.c (errors_flag): New variable.
    (set_warning_issued): Accept warning categories as an argument.
    * src/complain.h (Wall): Better definition.
    * src/getargs.c (flags_argmatch): Support for the new format.
    (usage): Update -Werror to -Werror[=CATEGORY] format.
    
    * src/complain.c (errors_flag): New variable.
    (set_warning_issued): Accept warning categories as an argument.
    * src/complain.h (Wall): Better definition.
    * src/getargs.c (flags_argmatch): Support for the new format.
    (usage): Update -Werror to -Werror=[CATEGORY] format.
    
    Signed-off-by: Akim Demaille <akim@lrde.epita.fr>

\end{verbatim}
\line(1,0){250}
\begin{verbatim}
commit bd52638008b361db9b368923a976a08d934efcd7
Author: Theophile Ranquet <ranquet@lrde.epita.fr>
Date:   Wed Sep 26 11:49:23 2012 +0200

    warnings: introduce -Wdeprecated in the usage info
    
    The deprecated warning, introduced some time ago, was not displayed in
    the usage message. This patch addresses the issue.
    
    * src/getargs.c (usage): Insert here.
    
    Signed-off-by: Akim Demaille <akim@lrde.epita.fr>

\end{verbatim}
\line(1,0){250}
\begin{verbatim}
commit b8e7ad588711ab1a87a7c30667a8d84fed72d75a
Author: Theophile Ranquet <ranquet@lrde.epita.fr>
Date:   Wed Sep 26 11:49:22 2012 +0200

    errors: prefix the output with "error: "
    
    This improves readability. This is also what gcc does.
    
    * NEWS: Document this change.
    * src/complain.c (complain_at): Prefix all errors with "error: ".
    (complain_at_indent, warn_at_indent): Do not prefix the context
    information of errors, which are basically just indented errors.
    * tests/conflicts.at, tests/glr-regression.at, tests/input.at,
    tests/named-refs.at, tests/output.at, tests/push.at,
    tests/regression.at, tests/skeletons.at: Apply this change.
    
    Signed-off-by: Akim Demaille <akim@lrde.epita.fr>

\end{verbatim}
\line(1,0){250}
\begin{verbatim}
commit a974c1ec2608ffbff4780210aa4bad025d361271
Author: Theophile Ranquet <ranquet@lrde.epita.fr>
Date:   Wed Sep 26 11:49:21 2012 +0200

    errors: indent "invalid value for %define" context
    
    This is the continuation of the work on the readability of errors
        context.
    
    For example, what used to be:
      input.y:1.9-29: invalid value for %define variable 'foo' : 'bar'
      input.y:1.9-29: accepted value: 'most'
    
    is now:
      input.y:1.9-29: invalid value for %define variable 'foo' : 'bar'
      input.y:1.9-29:     accepted value: 'most'
    
    * src/muscle-tab.c (muscle_percent_define_check_values): Use
    complain_at_indent to output with increased indentation level.
    * tests/input:at: Apply this change.
    
    Signed-off-by: Akim Demaille <akim@lrde.epita.fr>

\end{verbatim}
\line(1,0){250}
\begin{verbatim}
commit 6b1e1872d482435960c10f425df3c7000dc05266
Author: Theophile Ranquet <ranquet@lrde.epita.fr>
Date:   Wed Sep 26 11:49:20 2012 +0200

    errors: indent "%define var" redefinition context
    
    This is the continuation of the work on the readability of errors
    context.
    
    For example, what used to be:
      input.y:2.9-11: %define variable 'var' redefined
      input.y:1.9-11: previous definition
    
    is now:
      input.y:2.9-11: %define variable 'var' redefined
      input.y:1.9-11:     previous definition
    
    * src/muscle-tab.c (muscle_percent_define_insert): Use
    complain_at_indent to output with increased indentation level.
    * tests/input.at: Apply this change.
    
    Signed-off-by: Akim Demaille <akim@lrde.epita.fr>

\end{verbatim}
\line(1,0){250}
\begin{verbatim}
commit cbaea0106d2c70e45ddb5fa6d662d31c9bb3d4f8
Author: Theophile Ranquet <ranquet@lrde.epita.fr>
Date:   Wed Sep 26 11:49:19 2012 +0200

    errors: indent "symbol redeclaration" context
    
    This is the continuation of the work on the readability of errors
    context.
    
    For example, what used to be:
      input.y:5.10-24: %printer redeclaration for <field2>
      input.y:3.11-25: previous declaration
    
    is now:
      input.y:5.10-24: %printer redeclaration for <field2>
      input.y:3.11-25:     previous declaration
    
    * NEWS: Document this change.
    * src/symtab.c (symbol_redeclaration, semantic_type_redeclaration,
    user_token_number_redeclaration, default_tagged_destructor_set,
    default_tagless_destructor_set, default_tagged_printer_set,
    default_tagless_printer_set): Use complain_at_indent to
    output with increased indentation level.
    * tests/input.at: Apply this change.
    
    Signed-off-by: Akim Demaille <akim@lrde.epita.fr>

\end{verbatim}
\line(1,0){250}
\begin{verbatim}
commit 24d96dd3ebb49d003c29904fc5152ee5ec220e86
Author: Theophile Ranquet <ranquet@lrde.epita.fr>
Date:   Wed Sep 26 11:49:18 2012 +0200

    errors: indent "result type clash" error context
    
    This used to be the format of the error report:
    
      input.y:6.5-10: result type clash on merge function 'merge': [...]
      input.y:2.4-9: previous declaration
    
    In order to distinguish the actual error from the context provided, we
    rather this new output:
    
      input.y:6.5-10: result type clash on merge function 'merge': [...]
      input.y:2.4-9:     previous declaration
    
    Another patch will introduce an "error: " prefix to all non-indented
    lines, giving yet better readability to the reports.
    
    * src/complain.h (SUB_INDENT): Move to here.
    * src/reader.c (record_merge_function_type): Use complain_at_indent to
    output with increased indentation level.
    * src/scan-code.l (SUB_INDENT): Remove from here.
    * tests/glr-regression.at: Apply this change.
    
    Signed-off-by: Akim Demaille <akim@lrde.epita.fr>

\end{verbatim}
\line(1,0){250}
\begin{verbatim}
commit 41511178a71ebaf81bde7ffc682c575537af45cb
Author: Theophile Ranquet <ranquet@lrde.epita.fr>
Date:   Fri Sep 21 05:04:43 2012 +0200

    new Werror report format fixed in a test
    
    * tests/input.at : replaced [-Wyacc] with [-Werror=yacc]
    
    todo: fix the other failed test of the suite, tests/conflicts.at:1554

\end{verbatim}
\line(1,0){250}
\begin{verbatim}
commit fd01e1d05ea3e627033d148b5400b99a18ac7ba3
Author: Theophile Ranquet <ranquet@lrde.epita.fr>
Date:   Fri Sep 21 05:00:32 2012 +0200

    made previous commit less hairy
    
    * src/getargs.c : here

\end{verbatim}
\line(1,0){250}
\begin{verbatim}
commit 981c53e257f1974854edc4f6ad0e88c7f18e2bea
Author: Theophile Ranquet <ranquet@lrde.epita.fr>
Date:   Thu Sep 20 12:21:28 2012 +0200

    introduced a GCC-like -Werror=type
    
    * src/complain.h : errors_flag variable
    * src/complain.c : actual stuff happens here
    * src/conflits.c : differentiated SR and RR conflicts
    * src/getargs.c : flags_argmatch recognizes the new -Werror format

\end{verbatim}

  \end{adjustwidth}

  \cleardoublepage
  \cleardoublepage
\end{document}

